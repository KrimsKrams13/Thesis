\documentclass[11pt]{article} % DOKUMENTKLASSE. Mulighederne inkl. article, report, book, memoir  %
                              % Geometri-pakke: Styrer bl.a. maginer                              %
\usepackage[a4paper, hmargin={2.8cm, 2.8cm}, vmargin={2.8cm, 2.8cm}]{geometry}                    %
\usepackage{amssymb}          % Matematiske tegn og skrifttyper, bl.a. \mathbb{}                  %
\usepackage{amsthm}           % Opsætning der gør sætninger og beviser nemmere, se amsthdoc.pdf   %
\usepackage[utf8]{inputenc}   % Lidt kodning så der ikke kommer problemer ved visse konverteringer%
\usepackage[T1]{fontenc}      % Lidt kodning så der ikke kommer problemer ved visse konverteringer%
\usepackage{amsmath}          % Matematiske tegn                                                  %
\usepackage[colorlinks=true]{hyperref}   % Referencer som hyperlinks                              %
%\usepackage{txfonts}         % Times font                                                        %
\usepackage{graphicx}         % Graphicx pakke                                                    %
\usepackage{xparse}         % Graphicx pakke                                                    %
\usepackage{bigfoot}          % \verb in footnote                                                    %
\usepackage{caption}          % Caption-pakke                                                     %
\usepackage{subcaption}       % Subcaption-pakke                                                  %
\setlength{\captionmargin}{20pt}% Sætter caption-margin                                           %
\usepackage{listings}         % Listings. Indsætter kildekode pænt.                               %
\usepackage{float}            % Keeping figures in place                                          %
\lstset{                      %                                                                   %
  basicstyle=\ttfamily\footnotesize, % Sætter basisstilen i listings                              %
  showstringspaces=false,     % no special string spaces                                          %
  extendedchars=true,         % Bogstaver som æ, ø og å 
}                             %                                                                   %
\usepackage{verbatim}
\usepackage{enumitem}
\usepackage{courier}          % Courierskrifttype. Slankere skrivemaskineskrift i verbatim og     %
                              % listings                                                          %
\usepackage{multicol}         % Kolonner                                                          %
\usepackage[babel, titelside, nat, farve, en]{ku-forside}   % KU-forside med logoer               %
\def\HyperLinks{              % Hyperlinks-pakke, referencer til links og tillader links til www  %
\usepackage[pdftitle={\TITEL},pdfauthor={\FORFATTER}, % Et lille trick så pakken indlæses efter   %
pdfsubject={\UNDERTITEL}, linkbordercolor={0.8 0.8 0.8}]{}} %  titlen defineres.          %
\DeclareSymbolFont{rsfscript}{OMS}{rsfs}{m}{n} % Henter alfabet                                   %
\DeclareSymbolFontAlphabet{\mcal}{rsfscript} % Kalder dette alfabet for \mcal                     %
\setlength\arraycolsep{2 pt}  % Sætter kolonneafstanden i tabeller og eqnarrays til 2pt           %
\setcounter{tocdepth}{3}      % Dybde af indholdsfortegnelsen,                                    %
                              % 1 resulterer i kun sections kommer med, 2 er secs og subsec. etc  %
% \setcounter{secnumdepth}{0}   % Dybde af sektionsnummereringen.                                   %
%%%%%%%%%%%%%%%%%%%%%%%%%%%%%%%%%%%%%%%%%%%%%%%%%%%%%%%%%%%%%%%%%%%%%%%%%%%%%%%%%%%%%%%%%%%%%%%%%%%
\titel{Evaluating scalability of hash-based multi-core in-memory key-value stores} %                 %
\undertitel{Faculty of Computer Science} %                                                        %
\opgave{Thesis Report}                   % Findes kun under 'titelside' muligheden i ku-forside   %
\forfatter{Jakob Hanehoj}                %                                404                        %
\dato{\today}                            %                                                        %
\vejleder{Marcos Antonio Vaz Salles and Vivek Shah} % Findes kun under 'titelside' muligheden i KU-forside   %
\HyperLinks                              % Sætter pdf-titel til at svare til de just definerede   %
                                         %                                                        %
\newsavebox{\fminipagebox}
\NewDocumentEnvironment{fminipage}{m O{\fboxsep}}
 {\noindent\begin{lrbox}{\fminipagebox}
  \begin{minipage}{#1}\ignorespaces}
 {\end{minipage}\end{lrbox}%
  \noindent\fcolorbox{black}{used_grey}{\makebox[#1]{%
    \kern\dimexpr-\fboxsep-\fboxrule\relax%
    \usebox{\fminipagebox}%
    \kern\dimexpr-\fboxsep-\fboxrule\relax%
  }}%
 }
\definecolor{used_grey}{rgb}{0.8, 0.8, 0.8}
\definecolor{used_blue}{rgb}{0.8, 0.8, 0.2}
\definecolor{used_red}{rgb}{0.2, 0.8, 0.4}
\newsavebox{\globalminipagebox}
\NewDocumentEnvironment{globalminipage}{m O{\fboxsep}}
 {\par\kern#2\noindent\begin{lrbox}{\globalminipagebox}
  \begin{minipage}{#1}\ignorespaces}
 {\end{minipage}\end{lrbox}%
  \noindent\fcolorbox{used_red}{used_red}{\makebox[#1]{%
    \usebox{\globalminipagebox}%
  }}\par\kern#2
 }
\newsavebox{\localminipagebox}
\NewDocumentEnvironment{localminipage}{m O{\fboxsep}}
 {\par\kern#2\noindent\begin{lrbox}{\localminipagebox}
  \begin{minipage}{#1}\ignorespaces}
 {\end{minipage}\end{lrbox}%
  \noindent\fcolorbox{used_blue}{used_blue}{\makebox[#1]{%
    
    \usebox{\localminipagebox}%
    
  }}\par\kern#2
 }

\begin{document}                         %                                                        %
\maketitle % Laver titlen                %                                                        %
                                         %                                                        %
%%%%%%%%%%%%%%%%%%%%%%%%%%%%%%%%%%%% Indholdsfortegnelse %%%%%%%%%%%%%%%%%%%%%%%%%%%%%%%%%%%%%%%%%%

\begin{abstract}
\end{abstract}
\tableofcontents % indholdsfortegnelsen                                                           %
\listoffigures  % Liste over figurer \begin{figure} ... \end{figure}                              %
\listoftables   % Liste over tabeller \begin{table} ... \end{table}                               %
%%%%%%%%%%%%%%%%%%%%%%%%%%%%%%%%%%%%%%%%%%%%%%%%%%%%%%%%%%%%%%%%%%%%%%%%%%%%%%%%%%%%%%%%%%%%%%%%%%%

\newpage
\section{Introduction}

\newpage
\section{Background}
\label{sec:background}
In recent years there has been a lot of research in the field of hashing algorithms and key-value stores, including hash indices. We will first present required background information on key-value stores in Section~\ref{subsec:background_key_value_store}, then we will present required background information on hash functions in Section~\ref{subsec:background_hashing_algorithms}, and finally we will present required background information on hash indices in Section~\ref{subsec:background_hashing_indices}.
\subsection{Key-value stores}
\label{subsec:background_key_value_store}
A key-value store is a data storage pattern based on associative arrays, used to store, retrieve and manage data records. The data records may contain multiple fields of data, and are identified by a single unique key. The key value system interprets the value as a single collection of data, allowing different records to have different amounts of fields. This eliminates placeholders for optional values, which thus makes a key value more flexible and memory efficient than a traditional relational database.\\

Key-value stores adhere to the basic interface of persistent storage, \verb|CRUD|, defining the following four operations:
\begin{itemize}
  \item \verb|C|reate, for inserting a record in the store
  \item \verb|R|ead, for retrieving a record from the store
  \item \verb|U|pdate, for editing existing values in the store
  \item \verb|D|elete, for removing a record from the store
\end{itemize}
Many well known systems rely on this pattern, where some of the most well known are Cassandra~\cite{cassandra}, Redis~\cite{redis}, Oracle NoSQL Database~\cite{oracle}, HBase~\cite{hbase} and Dynamo~\cite{dynamo}. Recently much research have been done in the field of in-memory multi-core setups. Two of the most interesting results are the Silo~\cite{SILO13} and Masstree~\cite{MT12} databases. \\

Silo provides an ordered persistent data storage model, while being designed to use system memory and caches efficiently. This is achieved by avoiding all centralized points of contention. They also provide a commit protocol based on optimistic concurrency control (OCC), in which they employ \emph{ephocs}, in order to provide serializability while avoid all shared-memory writes of lookups. \\

Masstree also provides an ordered persistent flexible data storage model, which performs well on workloads in which many keys share a common prefix, due to the trie-like structure. It uses achieves fast concurrent operations using a scheme, in which lookups do no write shared data structures, thus not invalidating shared cache lines, achieved by optimistic concurrency control (OCC). Additionally, the lookups can be done in parallel with most inserts and updates, as writes only acquire local locks on the involved nodes, hence also allowing writes to different parts of the tree to happen concurrently. \\

Thus, a large amount of work has been put into combining interesting properties for ordered stores. We are, however, not aware of any system, which combines these properties in an unordered store. \\

\subsection{Hashing algorithms}
\label{subsec:background_hashing_algorithms}
The space of hashing algorithms is vast, spanning over different ideas for mapping arbitrary data into fixed sized data. In this section, hashing algorithms relevant for the project will be presented. We will start by presenting literature on the relevant hashing algorithms, and then go in details of each of the hashing algorithms.
\subsubsection{Overview of hashing algorithms.}
\label{subsubsec:background_review_of_hashing_algorithms}
In our selection process of hashing algorithms for this project, certain theoretical properties have been important. Firstly, the hashing algorithm should be able to handle variable length keys, as such keys are in focus in this project. Hashing algorithms designed for fixed length keys could be adapted to handle arbitrary length keys, by padding the keys with 0's. We have, however, chosen to focus on algorithms which are designed to be able to handle variable length keys. Secondly, the hashing algorithm should be robust to skewed input distribution, yielding a close to uniform output distribution regardless of the input distribution. Lastly, since the project focuses on performance scalability, the computational complexity of the hashing algorithm should be lowest possible.\\

One such hashing algorithm is found in simple tabulation hashing, which handles variable length keys and is simple both in terms of computational complexity and implementation, while being very robust. The tabulation hashing algorithm is by no means a new form of hashing, as the first instances of tabulation hashing was published in 1970 by Albert Zobrist~\cite{Zobrist}, and the method was later extended in greater generality by Carter \& Wegman in 1979~\cite{WC79}. \\

Simple tabulation hashing is 3-independent, while better implementations of tabulation-based methods can supply 5-independence~\cite{TZ09}. Pătraşcu and Thorup have recently shown that the low independence of simple tabulation hashing can be shown to provide many of the guarantees provided by higher independence~\cite{PT11}, yielding strong statistical properties for a very simple hashing method. Additionally, they have proven that tabulation hashing with linear probing can achieve $O(1)$ complexity for insertions, deletions, and lookups. Overall, this makes tabulation hashing interesting for this project.\\

The interest in tabulation hashing even more emphasized by a recent thorough comparison of hashing algorithms, performed by Richter et al.~\cite{RAD15}, in which they evaluate four hash functions, namely
\begin{itemize}
\item Multiply-shift hashing
\item Multiply-add-shift hashing
\item Murmur hashing
\item Tabulation hashing
\end{itemize}

Since multiply-shift hashing and multiply-add-shift hashing only have implementations for fixed-length keys, we have chosen to focus on murmur hashing and tabulation hashing. Richter et al. conclude that both of these hashing methods are slower than both multiply-shift hashing and multiply-add-shift hashing, even though both have greater robustness in terms of distributional properties, when faced with skewed input distributions. Between murmur hashing and tabulation hashing, they conclude murmur hashing to be the faster, with tabulation hashing being the most robust. \\

Murmur hashing is one of the most commonly used general purpose hash function, which is composed of several multiplications (MU) and rotations (R), thus creating the name (MU)(R)(MU)(R). Tabulation hashing and murmur hashing has therefore been chosen as the hashing algorithms evaluated in this project. We are not aware of any formal analysis of the algorithm, so we are using it as it is~\cite{Mur3}.

If the following two sections, simple tabulation hashing and murmur hashing will be presented.
\subsubsection{Simple tabulation hashing}
\label{subsubsec:background_simple_tabulation_hashing}
Tabulation hashing is a hashing scheme that combines table lookups and \verb|xor| operations in order to calculate the hash value, an integer type of a chosen size. It views a key \emph{x} as a vector of \emph{c} same size characters $x_1, ..., x_c$, and relies on one table $T_i$ for each of the \emph{c} character positions. The tables contain truly random\footnote{E.g. drawn from random.org, as suggested by Pătraşcu and Thorup} integer values, of the size of the wanted hash value. The hash function simply performs a lookup for each character in the corresponding table, and \verb|xor| the lookup results together:
$$h(x) = T_1[x_1] \oplus ... \oplus T_c[x_c]$$
The lookups are done by interpreting each character as an integer and using this integer to index into the corresponding table. The tables does therefore contain $2^{sizeof(character)}$ entries, and are initialized prior to execution of the hash function. This makes the complexity of the algorithm only depend on the speed of the table-lookups and the \verb|xor|'ing. By constructing the table small enough to fit in fast cache, the table lookups become very efficient, making the algorithm very fast while also being straightforward to implement.\\

The size of the characters influences both the space used and the runtime of the algorithm. If drawing keys from a universe of size \emph{u}, we see that we have $O(u^{1/c})$ entries in each table, making the total amount of entries become $O(cu^{1/c})$.\footnote{Example: Drawing a key of from the 128 bit universe \emph{u} and using a character size 8 bits yields $128/8=16$ characters, thus also 16 tables. Each of the tables will have to contain entries for the $2^8$ different combinations of a given character, thus containing $2^8 = 2^{128/16} = (2^{128})^{1/16} = O(u^{1/c})$ entries in each table. For \emph{c} characters we use \emph{c} tables, thus a total amount of entries becomes $O(cu^{1/c})$.} Since the \emph{c} in the exponent outweighs the factor \emph{c}, the total space required for the tables is decreased as the amount of characters is increased. Thus, in order to ensure that the tables can fit in fast cache (optimally \verb|L1|-cache), the size of the character could be decreased. \\

However, decreasing the size of the character yields more characters, which in turn yields more lookups and more \verb|xor| operations, thus increasing the runtime of the algorithm. The decision of a good character size is therefore a trade-off between having small enough characters for the tables to fit in fast cache, while not having too small characters to avoid too many computations.\\

An example of a hashing could be as following:\\ Assuming we get a key of size six bits (i.e. \verb|101100|) and use a character size of two bit, we get three characters, thus also three tables, each with four entries. The hash value type has been defined as an eight bit integer, meaning that each all the values in the tables will have to be eight bit integers. Three tables have been defined as seen in Figure \ref{fig:tabulation_tables}:\\
\begin{figure}
  \quad\quad\quad\quad\quad
  \begin{tabular}{|c|c|}
  \hline
  \multicolumn{2}{|c|}{Table 1}\\
  \hline
  00&01010110\\
  01&11010011\\
  10&00111011\\
  11&10101101\\
  \hline
  \end{tabular}
  \quad\quad\quad\quad
  \begin{tabular}{|c|c|}
  \hline
  \multicolumn{2}{|c|}{Table 2}\\
  \hline
  00&11001100\\
  01&00111001\\
  10&11011001\\
  11&11000101\\
  \hline
  \end{tabular}
  \quad\quad\quad\quad
  \begin{tabular}{|c|c|}
  \hline
  \multicolumn{2}{|c|}{Table 3}\\
  \hline
  00&10100111\\
  01&00101100\\
  10&01011001\\
  11&01010101\\
  \hline
  \end{tabular}
  \caption{Example of tabulation tables}
  \label{fig:tabulation_tables} 
\end{figure}
The hash function is then found by splitting the key into the three characters: $$x_1 = 10, x_2 = 11, x_3 = 00$$ Using these characters as index into their corresponding tables yield: $$T_1[x_1] = 00111011,\ \ \ \ T_2[x_2] = 11000101,\ \ \ \ T_3[x_3] = 10100111$$ Finally, \verb|xor|'ing these together yield the hash value: $$00111011 \oplus 11000101 \oplus 10100111 = 01011001$$
\subsubsection{Murmur hashing}
\label{subsubsec:murmur_hashing}
Murmur hashing is based on multiplications and rotations. It takes a key of any size, and splits it up into sections of the wanted hash value size. Currently, implementations for 32bit and 128bit hash values exist, causing sections of either 32bit or 128bit to be used. For each of these sections multiplications with statically defined values and rotations are performed. The rotations are essentially circular shifts, where the \verb|r| most significant bits are made the least significant bits, while the remaining bits are shifted to be the most significant bits.
\\
\begin{lstlisting}[frame=single]  % Start your code-block

  int rotl32 (int x, int r ) {
    return (x << r) | (x >> (32 - r));
  }
\end{lstlisting}
Finally, any remainder of the key which doesn't fit into exact section (i.e. the tail of the key) is included by a the murmur finisher~\cite{mur3}2, using shifts and multiplications with statically defined values.\\

Thus, the algorithm only uses $O(1)$ operations, yielding a low computational complexity, while also yielding  good distributional properties~\cite{RAD15}.
\subsection{Hash indices}
\label{subsec:background_hashing_indices}
Hash indices are fundamental data structures in computer science. Like key-value stores, hash indices always expose the basic four operations of \verb|get|, \verb|insert|, \verb|update| and \verb|remove|. Additionally, some implementations also expose \verb|range scans|. \\

In the following sections, we will first present the relevant literature on hash indices, then present the details on the chosen hash indices individually, and finally briefly summarize the key concepts of radix partitioning, which is needed for partitioning of hash indices, described in section \ref{subsec:_design_partition_hash_index}

\subsubsection{Review of hash indices}
\label{subsubsec:background_review_of_hash_indices}
Hash indices distribute data entries among a set of slots, while relying on a scheme for solving collisions in the slots. One of the simplest and most effective collision resolution schemes is chained hashing, in which collisions are resolved by placing all data entries of the same slot in some container, or chain. A common implementation of this strategy uses linked lists, as they are simple to implement and maintain. In this case, all operations can be solved by following or modifying pointers. \\

In the survey made by Richter et al.~\cite{RAD15}, they also include an analysis of five different hash indices, namely
\begin{itemize}
  \item Chained hashing
  \item Linear probing
  \item Quadratic probing
  \item Robin Hood Hashing on LP
  \item Cuckoo Hashing
\end{itemize}
They have shown that chained hashing is very resilient to unsuccessful queries, while providing overall decent average case performance. One of the reasons that the performance of chained hashing is worse than other hash indices under certain circumstances is the poor cache utilization using linked lists yields. This is due to new memory areas that will have to be loaded into the cache repeatedly. Additionally, the random memory jumps makes data-prefetching difficult for the hardware. For in-memory uses, these two disadvantages become quite significant. \\

Askitis and Zobel have addressed this issue by replacing the linked lists with dynamic arrays, in what they called array hashing \cite{NA09, AJ05}. They have shown that the obvious disadvantage of resizing the arrays dynamically is heavily outweighed by the advantages of better cache utilization and data prefetching in an in-memory setting. Additionally, using an array instead of linked lists yields better space utilization, as the overhead from nodes and pointers is eliminated. \\

In their most recent study, Askitis and Zobel specifically addressed implementing this version of chained hashing for integer-type hash values, and compared it to linear probing, bucketized cuckoo hashing and clustered chained hashing.\footnote{Bucketized cuckoo hashing and clustered chained hashing are both adaptations of their regular implementations, which try to improve cache utilization by placing a given amount of entries in a local memory region (bucket or cluster), thus offering some of the same advantages of array hashing over their regular counterparts} \\

Array hashing was found to be generally faster than bucketized cuckoo hashing and clustered chained hashing for all operations~\cite{NA09}. Linear probing was found to be the fastest in all but one situation; when the allocated data structure is sufficiently large. The requisite data structure size is however often difficult to estimate, which either leads to excessive space usage, making linear probing very inflexible, or exhibits bad performance. Array hashing is able to efficiently scale for greater loads. \\
\\
Finally, Dudás and Juhász have shown great multi-core scalability results for array hashing, using a lock-free structure~\cite{ADSJ13}. We have therefore found array hashing promising, and chosen to work on this structure. 
\\

A dual to the chained hashing is found in extendible hashing~\cite{dms03}, in which the buckets have a fixed size, while the directory grows when overflows happen in a bucket. This yields an upper limit on the probing time for all of the four operations (excluding range scans), which we've found interesting. Therefore, we will also be working with extendible hashing.
\\

\subsubsection{Chained hashing using arrays}
\label{subsubsec:background_chained_hashing_using_arrays}
The chained hashing data structure is based on the concept of a directory of pointers to buckets. The buckets are dynamically sized containers of data entries, and each data entry only contains its key and value.
The size of the directory is always fixed, meaning that there will not be any instantiation of additional buckets. Instead, chained hashing extends the individual buckets as they fill up. \\

In array hashing, the bucket containers are implemented using arrays. Due to the statically allocated size of arrays, this implementation needs to handle insertion and deletion more carefully, which causes a performance drawback, compared to the linked list implementation. The gain on the other hand comes from the efficient searching capabilities of the arrays, as well as better cache utilization and data prefetching.\\

The predefined size of the directory is always chosen to be a power of 2, \verb|dir_depth|, since this means that all the buckets can be enumerated using a given amount of bits.\footnote{Example: Defining the size of the directory to be of size $2^{16}$, all buckets in the directory can be enumerated using 16 bits} Thus, all of the operations (except the range scans) have in common that they locate the correct bucket for the key, by hashing the key to its hash value and looking at \verb|dir_depth| least significant bits of this hash value. \\

Common for the four operations \verb|get|, \verb|update|, \verb|insert| and \verb|remove| is that they all find the pointer to the bucket in which the operation is to be performed by hashing the key using the hash function, and then use the \verb|dir_depth| least significant bits of the hash value as an index into the directory. Once the correct bucket has been found, the individual operations proceed as described in the following paragraphs.

\paragraph{Get.} Once the correct bucket has been located, the lookup is performed by iteratively comparing the key of each data entry inside the bucket with the given key. If a match is found, the value of the data entry is returned and the lookup succeeds; otherwise, the next key is compared. If the end of a bucket is reached without finding a matching key, the lookup fails. \\
\\
\begin{fminipage}{\linewidth}
\begin{lstlisting}  % Start your code-block

get(key, value) {
  hash_value = hash_func.get_hash(key);
  bucket_index = hash_value & (dir_size-1); // Finds dir_depth LSB
  bucket = directory[bucket_index];
  for (int i = 0; i < bucket.keys.size(); i++) {
    if (key == bucket.keys[i]) {
      value = bucket.values[i];
      return;
} } }
\end{lstlisting}
\end{fminipage}

\paragraph{Update.} Updates are performed similarly to lookups, except that instead of returning the value of the correct data entry (if found), the value is updated to the new value. \\
\\
\begin{fminipage}{\linewidth}
\begin{lstlisting}  % Start your code-block

update(key, value) {
  hash_value = hash_func.get_hash(key);
  bucket_index = hash_value & (dir_size-1); // Finds dir_depth LSB
  bucket = directory[bucket_index];
  for (int i = 0; i < bucket.keys.size(); i++) {
    if (key == bucket.keys[i]) {
      bucket.values[i] = value;
      return;
} } }
\end{lstlisting}
\end{fminipage}
\paragraph{Insert.} Insertion is performed by locating the correct bucket, and placing the data entry at the first available slot. As for all separate chaining hashing variants, insertion into a full bucket is done by increasing the size of the bucket, thus here we increase the size of the array, which in turn might require a reallocation of the array. It is common practice to double the size of the array whenever an increase in size is required, as this eliminates situations where the array grows repeatedly, causing many expensive reallocation operations to be done in succession~\cite{DPH90}. \\

However, Askitis suggests that singular increases of the arrays are used, as they have found no significant performance increase when compared with size-doubling~\cite{NA09}. This effect is emphasized by Dudás and Juhász, who also increase the size by one every time, but do so by allocating a new array of size one larger, in order to be able to use CAS operations for synchronization~\cite{ADSJ13}. 

Using the approach of singular increase, the arrays will always be completely full, and thus not waste  memory. This does, however, also mean that any insert into the hash index will cause an increase of the size of a bucket, potentially causing many successive reallocation operations. \\
\\
\begin{fminipage}{\linewidth}
\begin{lstlisting}  % Start your code-block

insert(key, value) {
  hash_value = hash_func.get_hash(key);
  bucket_index = hash_value & (dir_size-1); // Finds dir_depth LSB
  bucket = directory[bucket_index];
  old_size = bucket.keys.size();
  bucket.keys.increment_size();             // Allocate space
  bucket.values.increment_size();           // Allocate space
  bucket.keys[old_size] = key;              // Insert the key
  bucket.values[old_size] = value;          // Insert the value
}
\end{lstlisting}
\end{fminipage}

\paragraph{Remove.} Removal is also done by finding the correct bucket in the directory, iteratively comparing the key of each data entries in the bucket with the given key, and removing the data entry if it is present. As for insertion, it is also suggested by Askitis that a removal decrements the size of the array, in order to avoid having empty slots in the array~\cite{NA09}. He did this by shifting all entries succeeding the found entry one slot towards the beginning, leaving the last slot empty to be removed. \\
\\
\begin{fminipage}{\linewidth}
\begin{lstlisting}  % Start your code-block

remove(key) {
  hash_value = hash_func.get_hash(key);
  bucket_index = hash_value & (dir_size-1); // Finds dir_depth LSB
  bucket = directory[bucket_index];
  for (int i = 0; i < bucket.keys.size(); i++) {
    if (key == bucket.keys[i]) {
      bucket.keys[i].remove();              // Remove the key
      bucket.values[i].remove();            // Remove the value
      bucket.keys.pack();                   // Copy keys to pack array
      bucket.values.pack();                 // Copy values to pack array
      return;
} } }
\end{lstlisting}
\end{fminipage}
\paragraph{Range scans.} Range scans are handled quite differently from the other four operations, as they do not operate on singular buckets. Since array hashing is an unordered data structure, scans for ranges of keys are not easily supported. In order to find all keys in a given range, a complete search of the entire structure must be performed by running through all entries in the hash index, and comparing them to the given range. This procedure certainly finds all the correct entries, but might be very slow, especially when searching for small range. The computational complexity always is $O(entry\_amount * probe\_time)$, irrespective of the size of the range. \\
\\
\begin{fminipage}{\linewidth}
\begin{lstlisting}  % Start your code-block

range_scan(start_key, end_key) {
  result = {};                              // Empty container for results
  for (int j = 0; j < directory.size(); i++) {
    bucket = directory[j];
    for (int i = 0; i < bucket.keys.size(); i++) {
      if (bucket.keys[i] >= start_key && bucket.keys[i] <= end_key) {
        result.add({bucket.keys[i], bucket.values[i]}); // Include KV pair
  } } }
  return result.sorted();                         // Return sorted results
}

\end{lstlisting}
\end{fminipage}

In this case of small ranges, it might be more useful to simply rehash every key in the searched range to obtain its bucket location, and search for it in the given bucket. In this way, the complexity of the range scan becomes dependent on the size of the range, as it becomes \verb|O(range_length * (hash_time + probe_time))|. The probing time is always constant, which makes this solution good for small ranges, but for longer ranges, the increased time caused by the additional hashing might take over, causing this to be worse than the full search. This procedure does also require for the key-type to be enumerable, as one would have to iterate over all keys between the start- and end-key. As we are using strings, this is not the case, making this solution unfeasible.

\subsubsection{Extendible hashing}
\label{subsubsec:background_extendible_hashing}
A dual approach to array hashing is found in extendible hashing~\cite{dms03}, which uses the same key components as array hashing, namely a directory pointing to buckets, which in turn hold the data entries. However, in extendible hashing, the directory can be seen as a dynamically sized array, which from start-up is quite small, but which will grow when needed.\footnote{Some implementations also allow the directory to shrink under certain circumstances.} The buckets of extendible hashing are fixed-sized containers for the data entries, and all buckets have the same size. \\

To describe the current size of the directory, a \verb|global_depth| denotes how many bits are needed to enumerate every bucket pointer in the directory. The maximum amount of buckets in the directory is therefore always $2^{(global\_depth-1)}$. Thus, to find the correct bucket for a given hash value, one follows the pointer whose position in the directory is equal to the \verb|global_depth| least significant bits of the hash value.\\

When the directory has to grow (more on when this happens later), we simply double its size, increment the \verb|global_depth|, and create a copy of every bucket pointer, stored in the newly allocated second half of the directory. The increase in \verb|global_depth| includes another least significant bit of the hash value, which is then used to distinguish between the original pointers and the newly copied pointers, where a 0 yields the old pointer, and a 1 yields the copy. Note that this directory increase does not create any new buckets, but simply allows for twice as many buckets to be stored when needed. We therefore end up with (at least) two pointers to each bucket, directly after the doubling of the directory. This directory doubling can be seen in Figure~\ref{fig:global_depth_increase}.\\

\begin{figure}[H]
  \centering
  \includegraphics[width=0.8\textwidth]{Graphs/global_depth_increase.png}
  \caption{Doubling of the directory for extendible hashing}
  \label{fig:global_depth_increase}
\end{figure}
In Figure~\ref{fig:global_depth_increase}, we see the directory in blue, originally with a \verb|global_depth| of two, thus containing four entries. Additionally, we have four buckets in red (A, B, C, D).\footnote{The buckets has been color coded, and all pointers pointing to the given bucket has been given the same color, for ease of understanding.} After the growth of the directory, the \verb|global_depth| is increased to three, which causes twice as many bucket pointers to be present. Each new pointer is a copy of an old one, thus pointing to the same bucket (indicated with dashed arrows of the same color). Notice that there has been no new allocations of buckets.\\

Thus, a bucket can be pointed to by multiple entries in the directory, as long as they have a common set of least significant bits. An individual \verb|local_depth|\footnote{The yellow boxes in Figure \ref{fig:global_depth_increase}} is stored in each bucket, which denotes how few bits of a hash value a bucket uniquely can be identified with. The \verb|global_depth| must always be kept at least as high as the \verb|local_depth| of any bucket, as the bucket would otherwise require more bits to be uniquely identified than what is used to identify buckets in the directory. \\

In Figure \ref{fig:global_depth_increase} it can be seen that all buckets have a \verb|local_depth| of two. On the After doubling the directoryall pointers going in to a given bucket shares the \verb|local_depth| least significant bits in the directory. \\

The difference between the \verb|global_depth| and a bucket \verb|b|'s \verb|local_depth| denotes how many pointers in the directory are pointing to \verb|b|, since this difference corresponds to the bits that, in addition to the least significant bits required to uniquely identify \verb|b|, are used to identify the pointer in the directory. Each combination of these \verb|global_depth - b.local_depth| bits yield a new pointer, which points to \verb|b|, and thus we have $2^{(global\_depth-b.local\_depth)}$ pointers pointing to \verb|b|. \\

These additional bits has been illustrated in Figure \ref{fig:global_vs_local}, indicated by a red box. 
\begin{figure}[H]
  \centering
  \includegraphics[width=0.7\textwidth]{Graphs/Global_vs_Local.png}
  \caption{Illustration of difference in global and local depth}
  \label{fig:global_vs_local}
\end{figure}
Since the difference in \verb|global_depth| and the \verb|local_depth| of each bucket is two, all buckets have $2^2 = 4$ pointers pointing to them. The four pointers of each bucket all have two common least significant bits, while the additional bits are unique for each pointer.\\

In extendible hashing, the pointer to the correct bucket for the four operations \verb|get|, \verb|insert|, \verb|update| and \verb|remove| is found by hashing the key using the hash function, and using the \verb|global_depth| least significant bits of the hash value as an index in the directory. Once the correct bucket has been found, the individual operations proceed as described in the following paragraphs. Also, since the extendible hashing primarily deviates from array hashing in the way it handles full buckets, only the two operations that change the amount of entries in the structure (namely \verb|insert| and \verb|remove|) are significantly different from array hashing implementation. Thus, the \verb|get|, \verb|update| and \verb|range-scan| operations are performed exactly as in array hashing, and are therefore not presented here.

\paragraph{Insert.} For inserts, an empty entry in the correct bucket is searched for. If such empty entry is present, the data entry is simply inserted there, and the insertion is finished. \\

However, when a data entry insertion is attempted into a full bucket, a collision occurs, which requires additional space to be found. This space is found by splitting the bucket into two, amongst which the entries in the original bucket and the new entry to be inserted are redistributed. To distinguish between the original bucket and the new 'image bucket', an additional least significant bit is used, such that hash values for which the additional bit is 0 remain in the old bucket, while hash values for which the additional bit is 1 are moved to the new bucket. This inclusion of an additional bit can simply be done by increasing the \verb|local_depth| of both buckets. By increasing the \verb|local_depth| of the original bucket, we also denote that fewer pointers should be pointing to the buckets, as only the pointers for whom their position in the directory has a (\verb|local_depth|+1)th least significant bit of 0, should be pointing to the original bucket, while the other half of the pointers should be pointing to the new image bucket.\\

An example of this splitting of buckets can be seen in Figure \ref{fig:bucket_split}.
\begin{figure}[H]
  \centering
  \includegraphics[width=0.8\textwidth]{Graphs/Bucket_split.png}
  \caption{Splitting of a full bucket}
  \label{fig:bucket_split}
\end{figure}
On this figure, we start by having four buckets, and all buckets have a capacity of four entries. Some entries have been inserted into the system, which are indicated by their hash values (i.e. the small numbers inside the boxes). Assume an insertion of a key with hash value 20 is attempted. Since the \verb|global_depth| is equal to three, to find the correct bucket pointer, we evaluate the three least significant bits of 20, which yields four. We then follow the pointer at index four in the directory, which points to bucket \verb|A|. Since bucket \verb|A| is full, we have to split the bucket, thus creating a new bucket \verb|A2|. For both the original bucket \verb|A| and the newly created 'image bucket' \verb|A2|, the \verb|local_depth| is increased to three. With the increased \verb|local_depth| the pointers previously pointing to bucket \verb|A| must be redistributed between \verb|A| and \verb|A2|, based on the newly included third bit. Next, we also reallocate all the entries in bucket \verb|A| between the two new buckets, by looking at the newly included third bit. For the entries with hash values 4 and 12, the third bit is set, and therefore these are placed in the the image bucket. For the entries with hash values 32 and 16, the third bit is 0, so these two entries stay in the original bucket. Finally, we retry the insertion of the entry with hash value 20, and see that there is an available spot in bucket \verb|A2|, which the pointer at index four now is pointing to. This concludes the split of the bucket.\\
\\
Thus, insertion into a full bucket \verb|b| with \verb|local_depth| \verb|ld| is done by the following procedure.\\
\\
\begin{fminipage}{\linewidth}
\begin{lstlisting}  % Start your code-block

  1 Create new empty bucket i, with same local_depth as b;
  2 Insert i in directory, (1<<ld) places after b.
  3 Increment b.local_depth and i.local_depth to ld+1;
  4 Split all pointers to b between b and i;
  5 For each data entry e in b:
    * Rehash e.key;
    * Insert e into the correct bucket, using the ld+1 least significant bits;
  6 Rehash the key of the new data entry, and find the correct bucket bn;
  7 If this bucket is full, start over from 1, using bn as b;
  8 Else Insert the new data entry into bn
\end{lstlisting}
\end{fminipage}
\vphantom{fill}\\
Note that in step 2 of the algorithm, the directory might not be large enough to contain the place where the image bucket is to be inserted. This is the case if and only if the \verb|global_depth| and \verb|local_depth| of the given bucket (prior to incrementing) is equal, and then the directory is simply doubled (as described previously). Also, when splitting a bucket, we have no guarantee that the entries in the original bucket will be split between the original and the new bucket. Specifically, if the newly included bit of all the entries is the same, the will all have to enter either the original or the new bucket, thus requiring an additional split of this bucket. The splitting of buckets can thus happen recursively, which is why the condition in step 7 is included.

\paragraph{Remove.} Removal of the entry is done exactly as for array hashing. However, several modifications to the bucket can then be applied afterwards, in order to reduce the probing time and memory footprint.

The simplest change is to ensure that all data entries in each bucket are always packed in the array, thus not leaving any empty data entries in between used data entries. This packing reduces the probing time, as an entry can be found not to be in a given bucket as soon as an empty entry is found. Since the order of the data entries in each bucket is not important, the packing can be achieved by moving the last entry in the bucket to the place, where the removed entry previously was.\\

Additionally, when removing data entries from a bucket, it might become possible to merge that bucket with the bucket, from which it was originally split, thus reducing the amount of buckets, leading to less memory used. This merging could be achieved by simply checking if there are sufficient space in the other bucket, and if this is the case, moving all the data entries as well as the pointer to the bucket. This approach would, however, increase the computational complexity of a removal operation, and could potentially cause additional splits when succeeding inserts are performed, thus also increasing the computational complexity of insert operations, while the only advantage is less memory used. We have therefore chosen not to implement merging of buckets.

This idea could however be taken even further, by checking if the directory could be shrunk to half its size. When merging two buckets as a result of a remove operation, one could always choose the bucket with a '0' as its most significant used bit, thus always placing the merged bucket in the first half of the directory. This approach could potentially lead to an empty second half of the directory, which could simply be removed, by decrementing the \verb|global_depth|. 

\subsubsection{Radix partitioning}
\label{subsubsec:background_radix_partitioning}
Since both array hashing and the extendible hashing are unordered key-value stores, in which range scans are handled by a full scan of the data-structure, these operations are expected to be inefficient. This inefficiency can be compensated for by partitioning the key space into a set of partitions, and creating a key-value store for each partition, each of which contains a subset of the key space. In order for this to improve the range scans, the partitioning of the keyspace must be sorted, such that all keys in one partition is greater than any key in the previous partitions, and lower than any key in the following partitions. \\
\\
One such partitioning is the radix partitioning of the key space, which partitions based on a prefix of the key. A common implementation of radix partitioning is seen in radix trees, as used by Leis et al. in their design of ARTful~\cite{ARTful}. A partitioned version of the hash indices using radix partitioning will be described in section \ref{subsec:_design_partition_hash_index}, while this section briefly introduces the concept of radix partitioning.\\

Radix partitioning is a partitioning based on a given set of most significant bits. These bits can be extracted using the radix function, which works by a shift and a logical \verb|AND| operation~\cite{radix}. To extract the bit range $[x, y)$ of a given key, one first right-shifts the key x places, and then perform an logical \verb|AND| between the shifted result and the constant $2^{y-x}-1$. The first part moves the x'th bit to become the least significant bit, while the second part generates a mask of $y-x-1$ 1's, which when logically \verb|AND|ed together with the result, yields the $y-x-1$ least significant bits. This yields a constant time partitioning function. Also, by using this radix function to extract a given amount of most significant bits, and using these bits for the partitioning, the partitions will become a sorting of the key space.
\newpage

\section{Design}
In this section, the primary design choices and the challenges involved with these choices will be presented and discussed. Rather than giving a chronological walk-through of the design of the system, this section will be issue-based, thus revolving around the design choices and challenges, rather than the structure of the implementation. This approach has been found preferable, since it makes it easier for a reader to read into a particular aspect of the design.\\
\\
First, the abstract interfaces and the guarantees given by the member functions is presented in Section~\ref{subsec:design_abstract_interfaces}. 
\subsection{Abstract interfaces}
\label{subsec:design_abstract_interfaces}
The abstract interface of the implemented hash functions will be briefly presented in Section~\ref{subsubsec:design_abstract_hash_function}. Next, a presentation and discussion of the abstract interface of the implemented hash indices is given in Section~\ref{subsubsec:design_abstract_hash_index}. Both sections include a presentation of the guarantees that have been adhered to by our implementations of the APIs.
\subsubsection{Abstract hash function interface}
\label{subsubsec:design_abstract_hash_function}
The abstract hash function interface defines the basic foundation of hash function implementations used. Since hash functions only has one purpose, namely to map a given key to a specific hash value, this interface only exposes one function, i.e. \verb|get_hash|. Since different hash functions yield different hash value types, the interface is templatized on this type, i.e. \verb|value_t|.
\begin{figure}[H]
  \centering
  \includegraphics[width=0.4\textwidth]{UML/abstract_hash_simplified.png}\\
  \caption{Class diagram for the abstract hash function interface.}\label{fig:UML_abstract_hash_function}
\newpage
\end{figure}
\noindent
The only guarantee of a hash function is determinism, meaning that for a given input key, the calculated output hash value must always be the same. Besides this guarantee, the property of uniformity (i.e. any input distribution yields an even spread of the output hash values) is desired but not required. To achieve this, there should not be a correlation between how close input keys are to each other, and how close the corresponding output hash values are to each other, as skewed input distributions otherwise would not yield uniform output distributions.
\subsubsection{Abstract hash index interface}
\label{subsubsec:design_abstract_hash_index}
The abstract interface of the implemented hash indices adhere to the general key-value store API known as \verb|CRUD|, as described in Section~\ref{subsec:background_key_value_store}, but extends the \verb|r|ead operation to both handle single key reads in the \verb|get| function, and multiple key reads in the \verb|range_scan| functions. The \verb|c|reate and \verb|d|elete operations are found in the \verb|insert| and \verb|remove| functions, respectively. The abstract interface can be seen in Figure~\ref{fig:UML_abstract_hash_index}.

\begin{figure}[H]
  \centering
  \includegraphics[width=0.8\textwidth]{UML/abstract_index_simplified.png}\\
  \caption{Class diagram for the abstract hash index interface.}\label{fig:UML_abstract_hash_index}
\newpage
\end{figure}
\noindent
As seen on the class diagram, the range scans have been split into two, namely \verb|range_scan| and \verb|reverse_range_scan|. This split is useful as the range scans guarantee ordered results, meaning that you can query for values in ascending or descending order.\\
\\
Since the hash indices are used in a multithreaded environment, atomicity guarantees are required. The four basic \verb|CRUD| operations all guarantee before-or-after atomicity, while the range scans only guarantee before-or-after atomicity of each individual value in the range. There are thus no guarantees regarding inclusion of values based on their order. Additionally, since the implemented system is not persistent, there are no all-or-nothing guarantees.

\subsection{Design choices on hash index operations}
\label{subsec:design_anything_interesting_on_the_operations}
The design of most of the operations for the hash indices have been trivial, since most of the operations simply access the correct bucket by indexing in the directory, iterate through the entries in the bucket and potentially perform an operation on one of the entries. Yet, both the \verb|remove| and the \verb|range_scan| of the operations have left design choices open, which will be described in this section. \\
\\
Since the design of the concurrency control scheme has been a large part of the project, everything regarding concurrency control will be left out of this section, and will instead be thoroughly discussed in Section~\ref{subsec:design_concurrency_control_scheme}. Also the design of the insert operation of extendible hashing has included many choices, but since the design of this operation has been heavily influenced by the concurrency control scheme, also this will be discussed in Section~\ref{subsec:design_concurrency_control_scheme}.
\paragraph{Design of the remove operation}
Probing through a bucket in search for whether or not a given key is present is part of the \verb|get|, \verb|update| and \verb|remove| operations, and is thus a core part of the majority of the operations. Hence, reducing the average probing time yields an increase in the performance of the operations. When naively removing an entry from a bucket, an empty slot will be left in the buckets key and value arrays. As suggested by Askitis~\cite{NA09}, empty slots in the arrays have been eliminated. Askitis does this by sliding all entries after the removed entry one position towards the start of the array. However, since we do not use the ordering in the arrays for anything, this has been changed to simply copying the last entry in the array to the position of the entry to be removed. This significantly reduces the amount of copying.

\paragraph{Design of the range scan operations using push operators}
As described in ~\ref{subsubsec:background_chained_hashing_using_arrays}, range scans are implemented as full scans of every entry in the data structure, searching for all entries with a key between a given \verb|start_key| and \verb|end_key|. The overall design of the full scans is trivial, since they are performed by iterating over every bucket in the directory, evaluating every entry in each bucket to see if its key is inside the given range. All entries inside the range are stored in a priority queue, using their key as the priority, which yields sorted access to the found results. \\
However, instead of returning every found entry at the end of the loop, we rely on the notion of push operators. A push operator is an object to which you can push a single result using the \verb|invoke| function. This function returns a boolean value, representing whether or not to stop pushing results. The implementation of the pushing operator is then responsible for handling the received data as intended. \\
\\
Thus, when all results of the range scan have been found and have been sorted, the push operator is iteratively invoked for each sorted result, as long as the \verb|invoke| function returns true. Using the push operators yield a high level of flexibility regarding the returned results, as this is left up to the client performing the queries. Additionally, the ability to preemptively stop the flow of return values enables more advanced range scan queries to be performed, such as finding the 100 keys following the \verb|start_key|. 

\begin{comment}
\subsection{Bucket designs}
\label{subsec:design_bucket_designs}
\end{comment}

\subsection{Cache and memory utilization}
\label{subsec:design_cache_and_memory_utilization}
Poor cache utilization generally hurts performance, as more more cache misses causes the CPU to access higher levels of cache or even RAM. Therefore, good cache utilization is important in all performance-intensive applications. For tabulation hashing, this is even more important, as the algorithm heavily relies on utilization of quick low level caches. The the memory utilization has therefore been an important design challenge.\\
\\
The primary memory usage comes from two components, namely the tabulation tables of tabulation hashing and data entries in the buckets of the hash indices.\\
\\
In order to ensure proper cache utilization of the tabulation tables, we ensure that no entry spans over two cache lines. This means that any give lookup in the tabulation tables at most will cause one cache line to be loaded into the cache. This property is achieved by ensuring that the tabulation table is cache line aligned, and that a certain whole amount of entries fill up a cache line. In this way, the entries will begin at the start of a given cache line, and every cache line will be completely filled with by a given amount of entries, thus no entries will span over multiple cache lines. \\
\\
Means have also been taken to ensure proper cache utilization of the buckets. First of all, the buckets have been cache line aligned. The buckets of both extendible hashing and array hashing uses 128B, and they do therefore span exactly two 64B cache lines or four 32B cache lines. Thus, these buckets will not cause unnecessary loads of cache lines. \\
\\
Additionally, the data entries in each bucket have been split to be stored in two arrays, one for keys and one for values. This was chosen since the majority of the operations (i.e. \verb|get|, \verb|update| and \verb|remove|) iterates through the keys until the searched key is found, and then needs access to the corresponding value. By splitting the keys and values into separate containers, iterating over keys only causes loading of the keys, thus avoiding loads of the corresponding values. Once the correct key is found, the corresponding value can be directly accessed.

\subsection{Concurrency control schemes}
\label{subsec:design_concurrency_control_scheme}
To ensure correctness of concurrent operations, concurrency control has been employed. The concurrency control scheme for array hashing and extendible hashing is quite similar, but with one significant difference. They have both been achieved with pessimistic concurrency control, using two kinds of locking primitives. In this section, we will first discuss the design of the used locking primitives in Section~\ref{subsubsec:design_locking_primitives}, and then present these two concurrency control schemes in sections~\ref{subsubsec:design_array_hashing_locking} and \ref{subsubsec:design_extendible_hashing_locking}.
\subsubsection{Locking primitives}
\label{subsubsec:design_locking_primitives}
Mutual-exclusion locks (mutexes) are popular synchronization primitives, primarily due to their immediate simplicity and thus ease of implementation. This makes mutexes fitted for the scope of this project. \\
\\
We have used two different locking primitives, which each has a set of advantages and disadvantages.\\
\\
The first locking primitive is a \verb|shared_mutex|, which protects shared data by only allowing threads with ownership of the mutex to access the data. The \verb|shared_mutex| has two levels of access, namely shared and exclusive access. Through shared access, several threads can share ownership of the mutex, while only one thread can own the mutex in exclusive mode. \verb|Shared_mutex|es are also referred to as \verb|readers/writer| locks, since they are commonly used when multiple threads are allowed to concurrently read from shared memory, while only one thread is allowed to write to the memory. This is commonly seen in access to in-memory data structures, making \verb|shared_mutex|es an obvious choice of locking primitive for this project.\\
\\
The primary advantage of \verb|shared_mutex|es comes from allowing shared access to the memory for non-modifying operations. These operations are thus allowed to run concurrently, while only operations that can cause conflicts (i.e. write operations) must be serialized. This does thus permit scalability of read operations, while providing the same correct execution of critical regions as regular mutexes. \\
\\
The \verb|shared_mutex|es also have some disadvantages. First of all, the cost of locking a \verb|shared_mutex| is higher than the cost of locking a plain mutex, since it contains more states to be handled correctly. Secondly, locking a \verb|shared_mutex| modifies the state of the mutex, both for for shared and exclusive accesses. Doing so repeatedly from multiple threads thus causes the cache-line containing this state to be transfered a lot, causing poor cache utilization. The smaller the protected critical regions are, the more significant this issue becomes, as a higher throughput over the critical regions can be achieved, causing more cache contention. Finally, when a thread is blocked from taking ownership of a mutex, a context switch usually happens, scheduling another thread to start.\\
\\
To prevent these disadvantages, we have also employed a second locking primitive building on \verb|spinlock|s~\cite{spinlock}. As opposed to \verb|shared_mutex|es, \verb|spinlock|s only provide serialized access to the protected resource, as they block any thread after the first by looping (spinning) until the lock becomes available. \\
\\
Spinlocks also have a set of advantages and disadvantages, compared to mutex implementations. First of all, the unlocking of a spinlock is done by a single atomic write, and does thus not enter kernel space to wake up other threads waiting for the same lock. This makes unlocking much more efficient, while also prevents cache pollution from entering kernel space. Additionally, since blocking on a spinlock does not put the thread to sleep, taking ownership of the lock once it becomes available can possibly be done with much lower latency, yielding higher serialized throughput. \\
\\
However, since only one thread can have ownership of a spinlock, this locking primitive does not yield any immediate concurrency. Concurrency will therefore have to be facilitated by the code in which the spinlock is used, e.g. by spreading the concurrent operations over different protected resources. Secondly, a spinlock does not in itself provide memory visibility, i.e. ensure that any writes performed in a critical region will be visible to succeeding threads. Memory visibility does thus have to be manually enforced. This has been achieved by \textcolor{red}{Either write about mutex, or verify that memory fenced can be properly used}.
\subsubsection{Array hashing locking}
\label{subsubsec:design_array_hashing_locking}
The locking scheme employed for array hashing is very simple. The directory of array hashing does not mutate after initial allocation, and does therefore not need any locking. This leaves two constructs that potentially could yield a level of locks each, namely the buckets and the data entries inside the buckets. Enabling concurrent access to multiple data entries inside a given bucket would require concurrent access inside the bucket. This would thus in turn require the bucket level locks to be implemented using \verb|shared_mutex|es. \\
\\
However, since the operations performed inside each bucket are small, while the contention on each bucket is expected to be low, the overhead of a \verb|shared_mutex| exceeds the performance increase obtained by allowing concurrent access to different entries inside the same bucket, as we shall later see (in Section~\ref{subsubsec:ycsb_results}). The locking scheme in the array hashing implementation does therefore use a single level of locks, namely one uncontested mutex for each bucket, yielding exclusive access to the given bucket. \\
Synchronization of the four \verb|CRUD| operations have all been implemented using strict two phase locking (S2PL). They all start by taking the lock for the bucket of the query, and keep this lock until the end of the operation. Using S2PL ensures before-or-after atomicity of these operations. \\
\\
An example of the locking scope can be seen on the following pseudo code for the read operation, in which the yellow lines indicate the scope of the lock:\footnote{Note: A 'Resource Acquisition Is Initialization' (RAII) scheme is assumed to be implemented by the \verb|lock| class in this example, meaning that the locks will unlock when they are destructed at the end of the scope.}\\
\begin{fminipage}{\linewidth}
\begin{lstlisting}[escapechar=@]
get(key, value) {
  hash_value = hash_func.get_hash(key);
  bucket_index = hash_value & (dir_size-1); // Finds dir_depth LSB  @\begin{localminipage}{0.95\textwidth}
.\ lock bucket\_lock = lock(bucket\_mutexes[bucket\_index]);\\
.\ bucket = directory[bucket\_index];\\
.\ for (int i = 0; i < bucket.keys.size(); i++) \{\\
.\ \ \ if (key == bucket.keys[i]) \{\\
.\ \ \ \ \ value = bucket.values[i];\\
.\ \ \ \ \ return;\\
.\ \} \}
  \end{localminipage}@
}
\end{lstlisting}
\end{fminipage}
\vphantom{fill}\\
For the range scans, we have chosen to keep the locks for as short a duration as possible, while still yielding before-or-after atomicity for each data entry. As mentioned previously (in Section~\ref{subsec:design_anything_interesting_on_the_operations}), the range scans iterate over all the buckets in the directory, and for each bucket iterate through all the entries. Each iteration over the buckets start by taking the lock for the given bucket and end by releasing the given lock. This approach ensures exclusive access to each bucket, and does thus yield before-or-after atomicity for entire buckets, thus fulfilling the atomicity guarantee. Due to the granularity of the locks, it is not possible to make an implementation that adhere to the atomicity guarantee in a tighter way.\\
\\
The lock scope of the range scan operation can be seen on the following pseudo code, again indicated by yellow lines:\\
\\
\begin{fminipage}{\linewidth}
\begin{lstlisting}[escapechar=@]
range_scan(start_key, end_key) {
  result = {};                              // Empty container for results
  for (int j = 0; j < directory.size(); i++) {
    bucket = directory[j];  @\begin{localminipage}{0.95\textwidth}
.   \ lock bucket\_lock = lock(bucket\_mutexes[j]);\\
.   \ for (int i = 0; i < bucket.keys.size(); i++) \{\\
.   \ \ \ if (bucket.keys[i] >= start\_key \&\& bucket.keys[i] <= end\_key) \{\\
.   \ \ \ \ \ result.add({bucket.keys[i], bucket.values[i]}); \\
.\ \ \ \ \ \}\\
.\ \ \ \}
  \end{localminipage}@
  }
  return result.sorted();                         // Return sorted results
}
\end{lstlisting}
\end{fminipage}
\vphantom{fill}\\
Since this locking scheme is not conservative, deadlocks could potentially be possible. This can, however, easily be seen not be the case, as the locking scheme for all of the operations never hold more than one lock at a time, thus eliminating any circular dependencies.
\subsubsection{Extendible hashing locking}
\label{subsubsec:design_extendible_hashing_locking}
The locking scheme for extendible hashing is very similar to the locking scheme for array hashing. The key difference is that the directory in extendible hashing does mutate after allocation, and does therefore also require a coarse grained lock on the directory, in addition to the lock on each individual bucket. \\
\\
The buckets of extendible hashing and the buckets of array hashing are very similar, except that the limit on the amount of entries in each bucket potentially reduces the contention in each bucket. Thus, by similar reasoning as for the array hashing implementation, employing a third fine grained level of locks on the individual entries in the buckets does not improve the scalability.\\
\\
A two level locking scheme has consequently been implemented. As for array hashing, the bucket locks of extendible hashing has been implemented using the spinlock implementation. However, since all operations require access to the directory, the directory lock will be heavily contented. Serializing the access to this lock will consequently cause a complete serialization of all operation. This locking level has therefore been implemented using \verb|shared_mutex|es, allowing operations that only require shared access to the directory to happen concurrently. \\
\\
Since we do not merge buckets when deleting entries, the only operation that modify the directory is \verb|insert|, which therefore is the only operation that require exclusive access to the directory lock. For all operations except for \verb|insert|s, the two level locking scheme does not cause a significant change to the code, as shared access to the directory lock simply is obtained at the start of the operations. An example of this can be seen in the following pseudo code, where green lines indicate the scope of the directory lock:\\
\\
\begin{fminipage}{\linewidth}
\begin{lstlisting}[escapechar=@]
get(key, value) {
  hash_value = hash_func.get_hash(key); @\begin{globalminipage}{0.95\textwidth}
.\ lock directory\_lock = lock\_shared(global\_shared\_mutex);\\
.\ bucket\_index = hash\_value \& (dir\_size-1); // Finds dir\_depth LSB \begin{localminipage}{0.95\textwidth}
.\ lock bucket\_lock = lock(bucket\_mutexes[bucket\_index]);\\
.\ bucket = directory[bucket\_index];\\
.\ for (int i = 0; i < bucket.keys.size(); i++) \{\\
.\ \ \ if (key == bucket.keys[i]) \{\\
.\ \ \ \ \ value = bucket.values[i];\\
.\ \ \ \ \ return;\\
.\ \} \}
  \end{localminipage}\end{globalminipage}@
}
\end{lstlisting}
\end{fminipage}
\vphantom{fill}\\
Note that the hash value calculation is performed before the taking of the lock. This simple modification reduces the critical region marginally, while it does not imply any concurrency issues, as the hash functions do not get modified.\\
\\
The directory lock does thus only imply significant changes to the implementation of the \verb|insert| operation. The \verb|insert| operation could also simply be implemented by taking the directory lock in exclusive mode at the start of the operation. This would be a correct implementation, which can easily be seen not to cause any deadlock, as the directory lock is always taken before any of the bucket locks, eliminating circular dependencies. However, as described in Section~\ref{subsec:design_anything_interesting_on_the_operations}, only a very limited amount of insert operations will cause a change to the directory. Therefore, taking the directory lock in exclusive mode for every insert operation would cause the majority of these operations to perform less than optimal. \\
\\
Two optimizations over this naive implementation has been identified and implemented. They both follow the same general principle of allowing most inserts to concurrently execute a 'fast path', while only making certain inserts go through a 'slow path'. The fast path should thus be valid to execute under shared access of the directory lock, while only the slow path requires exclusive access. The general idea is shown in the following algorithm: \\
\\
\begin{fminipage}{\linewidth}
\begin{lstlisting}[]
insert(key, value) {
1.  obtain shared access to directory lock
2.  calculate bucket number, bn

3.  obtain exclusive access to the given bucket lock 
4.  if fast path is possible:
5.    perform fast path and return
 
6.  obtain exclusive access to directory lock
7.  perform slow path and return
}
\end{lstlisting}
\end{fminipage}
\vphantom{fill}\\
A general complication regarding the fast path/slow path scheme is the transition from the fast path to the slow path. Any improvement over the naive locking implementation should ensure that the directory locks is always held, whenever a local bucket lock is to be taken, as a circular dependency otherwise could occur, potentially causing a deadlock. Thus, to make a fast path possible, shared access must be obtained before any interaction with the bucket locks happens. However, for some inserts the fast path will not be possible, causing them to enter the slow path (if condition at step 4. fails). Obtaining exclusive access at this point (step 5.) could potentially cause a circular dependency between the directory lock and the bucket lock. This dependency could be solved by releasing the bucket lock, but this could in turn invalidate reads performed while it was held.\\
\\
The solution to this complication is to completely abort the 'transaction' of the fast path, when it is realized that the slow path is required, and restart the lock acquisition phase, this time obtaining exclusive access to the directory lock. To make the abortion valid, every modification done under the fast path must be reverted. To make this as simple as possible, we have chosen to perform all the required checks for whether or not the fast path is possible before performing any modifications. Aborting is therefore as simple as releasing the held locks and starting over.\\
\\
\noindent
\begin{fminipage}{\linewidth}
\begin{lstlisting}[]
insert(key, value) {
1.  obtain shared directory lock access
2.  calculate bucket number, bn

3.  obtain exclusive bucket lock access
4.  if fast path is possible:
5.    perform fast path and return

6.  release bucket and directory lock
7.  obtain exclusive directory lock access
8.  perform slow path
}
\end{lstlisting}
\end{fminipage}
\vphantom{fill}\\
This provides the general framework for the improvements over the naive implementation. \\
\\
To understand why exactly two improvements can be made, an evaluation of the different types of inserts has to be made. The inserts can be classified into three groups, namely:
\begin{itemize}[noitemsep]
  \item Inserts into a bucket with an available slot, thus causing no splits at all.
  \item Inserts into a full bucket, thus causing a bucket split, which does not cause a directory doubling.
  \item Inserts into a full bucket, which does cause a directory doubling.
\end{itemize}
Since neither of the two first groups of inserts modify the directory, both of these can be put into the fast path, as seen by the following two improvements.\\
\\
The first improvement over the naive lock implementation allows inserts into a bucket containing an empty slot to happen under shared access to the directory lock. Using the presented framework this can easily be implemented: \\
\\
\begin{fminipage}{\linewidth}
\begin{lstlisting}[]
insert(key, value) {
1.  obtain shared directory lock access
2.  calculate bucket number, bn

3.  obtain exclusive bucket lock access
4.  if directory[bn] has empty spot:
5.    insert new entry and return

6.  release bucket and directory lock
7.  obtain exclusive directory lock access
8.  perform slow path
}
\end{lstlisting}
\end{fminipage}
\vphantom{fill}\\
Including the second group of inserts in the fast path requires a bit more analysis. \\
\\
As described in Section~\ref{subsec:background_hashing_indices}, splits of buckets can happen recursively. In order to avoid modifying anything before ensuring that the splitting of buckets does not cause a doubling of the directory, we have to calculate the maximum \verb|local_depth| of any bucket created by the insertion. If this \verb|local_depth| is less than or equal to the \verb|global_depth|, we know that the directory wont be doubled. \\
\\
To do this, first recall that the \verb|local_depth| of both the original and the new 'image' bucket is increased by one for every split, which corresponds to the additional bit of the entries' hash values, based on which the entries are redistributed. Since a split will happen for every additional bit all of the entries have in common, the \verb|local_depth| of the final bucket will be one greater than the amount of common least significan bits among all the entries' hash values, thus equal to the position of the first differing bit. \\
\\
The second improvement can thus be included in the implementation as following (including lock scopes):\\
\\
\begin{fminipage}{\linewidth}
\begin{lstlisting}[escapechar=@]
insert(key, value) {@\begin{globalminipage}{0.95\textwidth}
1.  obtain shared directory lock access\\
2.  calculate bucket number, bn \begin{localminipage}{0.95\textwidth}
3.  obtain exclusive bucket lock access\\
4.  if directory[bn] has empty spot:\\
5.    insert new entry and return\\
\\
6.  calculate maximum local depth, mld\\
7.  if mld <= global\_depth:\\
8. \ \ create new buckets, adjust pointers, redistribute old entries, \\
. \ \ \ \ \ insert new entry and return\\
\\
9.  release bucket and directory lock \end{localminipage}\end{globalminipage} \begin{globalminipage}{0.95\textwidth}
10.  obtain exclusive directory lock access\\
11.  perform slow path\end{globalminipage}@
}
\end{lstlisting}
\end{fminipage}
\vphantom{fill}\\
With these two improvements in place, we can see that only inserts that does cause a directory doubling to happen will enter the slow path. This allows us to estimate a limit on how many times the slow path can be taken: Each run through the slow path causes at least a doubling of the directory\footnote{Note: Since the buckets can split recursively, the \verb|local_depth| of a given bucket can become more than one greater than the \verb|global_depth|, requiring multiple doublings of the directory}, and therefore every such run will at least double the memory overhead from the bucket pointers. Thus, just 30 runs through the slow path will cause a minimal initial directory of one bucket to have a memory overhead from just the bucket pointers of 4.29GB, which is doubled by every following run, completely ignoring the memory used by the actual data. This will quickly hit a memory capacity of the machine, thus ensuring that a low amount of runs through the slow path will happen in any setup.\\
\\
One final thing to notice is that all the checks performed in the fast path becomes invalidated as soon as the bucket and directory locks are release. Therefore, these checks will also have to be performed in the slow path (i.e. redoing steps 2-8, except for step 3.), but since this path is taken so seldomly, redoing this will not be of any significance.
\subsection{Partitioning of hash indices}
\label{subsec:_design_partition_hash_index}
As mentioned in Section~\ref{subsubsec:background_radix_partitioning}, an attempt to compensate for the expected low performance of the range scan operations in the unordered hash indices, a partitioned version of the hash indices has been implemented. The idea behind partitioning the hash indices is to create sorted partitions of the key space, such that range scans can omit partitions that completely lie outside of the range. One way of creating such sorted partitions of the key space is by using radix partitioning, thus partitioning based on a fixed amount of most significant bits from every key. \\
\\
Partitioning of the key space does not change the semantics of any of the operations, except for range scans, since these operations can be performed inside the partition whose key space contains the data entry's key. Thus, to perform either of the single key \verb|CRUD| operations, one simply finds the correct partition by evaluating the most significant bits, and invokes the operation on the hash index of the corresponding partition.\\
\\
Since the range scans includes multiple keys, they can include keys from multiple partitions. This makes an improvement to the full range scan of the unordered hash indices possible, as described in the following section.
\paragraph{Range scans} All data entries in the range of the scan will be resident in one of the following three partition groups:
\begin{itemize}[noitemsep]
  \item The partition containing the start-key, namely the start-partition \verb|SP|;
  \item The partition containing the end-key, namely the end-partition \verb|EP|;
  \item Any of the partitions between the \verb|SP| and \verb|EP|, namely the middle-partitions \verb|MPS|.
\end{itemize}
A fourth group of partitions exist, namely the partitions handling parts of the key space, which is completely excluded from the range of the scan. Instead of a full scan of all the partitions, scanning the three former groups will suffice in order to ensure that all data entries in the range of the scan are included. All the searched data entries can thus be found as following:\\
\\
Firstly, all data entries in the \verb|SP| whose key is greater than or equal to the start-key can be found by invoking an open-ended range scan in the \verb|SP|. Secondly, all data entries in any partition between the \verb|SP| and the \verb|EP| can be found by invoking full-range scans on the \verb|MPS| partitions. Lastly, all data entries in the \verb|EP| whose key is less than or equal to the end-key can be found by invoking a range scan operation between the lowest possible key\footnote{Since our keys are strings, this corresponds to the empty string.} and the end-key in the \verb|EP|.\\
\\
This approach will potentially reduce the amount of keys to be searched, as all keys resident in excluded partitions are cut away, thus reducing the average case computational complexity of the algorithm. Additionally, distributing the data entries across partitions cause less contention on each partition, which potentially can reduce the probing time of \verb|get|, \verb|insert| and \verb|update| operations, which in turn increases their average performance.

\subsection{Hash function and hash index in client threads}
\label{subsec:design_all_in_threads}
\textcolor{red}{Move to experiments?}
\newpage
\section{Implementation}
\subsection{Hash functoins}
\label{subsec:implementation_hashing_algorithms}
In the following section, implementation-specific details about the hash functions is described. First, the abstract hash function interface is described in Section~\ref{subsubsec:implementation_abstract_hash_function}, followed by the descriptions of the tabulation hashing and murmur hashing implementations in sections~\ref{subsubsec:implementation_tabulation_hash} and~\ref{subsubsec:implementation_murmur_hash}, respectively.
\subsubsection{Abstract hash function}
\label{subsubsec:implementation_abstract_hash_function}
The abstract hash function interface has been implemented in the \verb=abstract_hash.h= file. It only exposes the virtual \verb|get_hash(std::string)| function, to be overritten. The class is templatized by the type of the hash values, defined as \verb|value_t|, yielding a generic interface for hash functions. However, in all of our hash function implementations, \verb|value_t| has been set to a 32bit unsigned integer. As we shall see in Section~\ref{subsubsec:implementation_tabulation_hash}, choosing a fixed hash value type has been required for tabulation hashing.
\subsubsection{Simple tabulation hashing}
\label{subsubsec:implementation_tabulation_hash}
The tabulation hashing algorithm has been implemented in the \verb=tabulation_hash.h= file. It contains the \verb|tabulation_hash| class, which extends the \verb|abstract_hash| class. In addition to the template parameter of \verb|abstract_hash|, the \verb|tabulation_hash| class is also templatized by the amount of tabulation tables to be generated, which is equal to the maximum key length. The class thus holds the tables needed for the hashing, and only exposes one method, namely \verb|get_hash(std::string)|, which calculates the hash value based on the input string, using the tabulation tables. \\
\\
\begin{fminipage}{\textwidth}
\begin{lstlisting}[]  % Start your code-block

  value_t tabulation_hash::get_hash(std::string key)
\end{lstlisting}
\end{fminipage}
\vphantom{fill}\\
In order to adhere to the suggestion by Pătraşcu and Thorup to draw the random values for the tabulation tables, the size of the these values has to be defined prior to execution. Additionally, since the size of the hash value is equal to the size of the tabulation table values, the hash value size also has to be defined prior to execution. The \verb|value_t| type has thus been set to \verb|uint32_t|. \\
\\
As described in Section~\ref{subsubsec:background_simple_tabulation_hashing}, the tabulation hashing algorithm also uses a fixed character size, which is a tradeoff between computational complexity and memory usage. Due to the following reasoning, we have been using a rather small character size of 1B: \\
By setting the character size to 1B, there will be 256 entries in each table. Since each entry contains the hash value for the given key, each entry will be of size 4B, making the total size of each table become \emph|256*4B = 1kB|. Thus, to be able to hash a string of e.g. length eight, the memory needed to hold the tables will be 8kB. By using character sizes of 2B, amount of tables needed would be reduced by a factor of 2, but it would increase the amount of entries in each table by a factor $2^{8}$, thus increasing the total size needed by a factor $2^7$. Thus, for the same string of length eight, the total memory required would become $2^{16}*(8/2)*4B$ = 1MB, making the tables too large for fast cache. Therefore, the character size has been fixed to 1B.\\
\\
As for the generation of the random values in the tables, the suggestion of taking truly random values from \verb|random.org| has been adhered to to some extend, as 16 tables of 256 entries each has been generated and hardcoded into the implementation, found in the \verb|random_number.h| file. When instantiating the \verb|tabulation_hash| class with 16 or less tabulation tables, only the required amount of these values are loaded into the used tabulation tables, thus not wasting unused memory. However, if more than 16 tables are needed, the tables are generated using the \verb|std::mt19937_64| implementation of the mersenne twister~\cite{MT02}.\\
\\
The actual hashing is done by finding the C-string representation of the key (i.e. a pointer to an array of characters, \verb|reinterpret_cast|ing this pointer to a \verb|std::uin8_t| pointer, and iterating over the \verb|key.size()| bytes following this pointers location, i.e. every character of the key. The \verb|std::uint8_t| representation of each character is then used as an index in the table corresponding to the given character. The results of all these lookups are simply \verb|xor|'ed together.

\subsubsection{Murmur hashing}
\label{subsubsec:implementation_murmur_hash}
The murmur hashing algorithm is implemented in the \verb|murmur_hash_32.h| file, containing the \verb|murmur_hash_32| class, which also extends the \verb|abstract_hash| class. As mentioned in Section~\ref{subsubsec:background_review_of_hashing_algorithms}, we are using the murmur hashing implementation provided by A. Applyby~\cite{Mur3} almost as it is. Two minor simplifications are as following:\\
\\
First of all, the provided implementation has functionality for generating both 32bit and 128bit hash values. Since we are using 32bit integers, we only used the 32bit implementation. Secondly, to adhere to the interface provided by the \verb|abstract_hash| class, the hash value should be calculated by the \verb|get_hash| function, which only takes a string as parameter. In the provided implementation, the parameter list includes a seed, which is used as a starting value for the calculations. Instead of providing this seed to the function every time, we instantiate it using the time since epoch when the class is instantiated. \\
\\
These simplifications merely make the implementation fit our needs, while not changing the semantics of the algorithm.

\subsection{Hash indices}
\label{subsec:implementation_hashing_indices}
In this section, implementation-specific details about the hash indices is described. First, the abstract hash index will be described in Section~\ref{subsubsec:implementation_abstract_index}. Next, the hash indices are described in sections~\ref{subsubsec:implementation_array_hashing}, \ref{subsubsec:implementation_extendible_hashing} and~\ref{subsubsec:implementation_partitioned_array_hashing}. Finally, the implementation of the locking primitives are described in Section~\ref{subsubsec:implementation_locking_primitives}.
\subsubsection{Abstract hash index}
\label{subsubsec:implementation_abstract_index}
The abstract hash index interface has been implemented in the \verb|abstract_index| class, which simply exposes the four \verb|CRUD| operations \verb|get|, \verb|insert|, \verb|update| and \verb|remove|, as well as both the ascending \verb|range_scan| and descending \verb|reverse_range_scan|. The class as well as the functions are purely virtual, and can thus not be instantiated.
\subsubsection{Chained hashing with arrays}
\label{subsubsec:implementation_array_hashing}
The array hashing has been implemented in the \verb|array_hash_table.h|file. It contains the \verb|array_hast_table| class, which extends the \verb|abstract_index| class. The class contains the directory as well as the hash buckets. It is templatized by the \verb|directory_size|, used for instantiating the directory. Finally, the class is instantiated with a hash function (i.e. an implementation of the \verb|abstract_hash| interface), which is used for all hashing calculations.\\
\\
The hash buckets has been implemented in the \verb|hash_bucket| struct, which contains the arrays for the keys and values, as well as the buckets lock.\\
The arrays are implemented using \verb|std::vector|s, which makes the dynamic incrementation of the size very simple, since inserting entries using the \verb|push_back| causes an expansion of the vector. Since the expansion policy is implementation specific, the single-element increases suggested by both Askitis~\cite{NA09} as well as Dudás and Juhász~\cite{ADSJ13} is not possible. When compiling with g++, the capacity of the arrays are doubled when an expansion is. However, since both of them only have reported equal performance for single-element increases, this has not been seen as an issue. \\Conversely, doubling the capacity of the array might be preferable, since we do not have an upper bound on the amount of entries in each bucket for array hashing, which might cause the buckets to become large. Expanding a vector usually causes reallocations of the entire vector, which includes copying all the elements. The larger the vector, the more expensive the reallocation cost is, since more elements have to be copied. The vectors have therefore been found suitable for implementing the buckets.\\
As described in Section~\ref{subsubsec:design_array_hashing_locking}, the bucket locks have been implemented using spinlocks. The implementation of the spinlocks will be described in~\ref{subsubsec:implementation_locking_primitives}.\\
\\
The \verb|directory| is implemented as a \verb|std::vector| of \verb|hash_bucket|s, which is instantiated with the wanted size, based on the \verb|directory_size| parameter.\\
\\
All of the four \verb|CRUD| operations finds the bucket to operate in by first hashing the key, and then using the least significant bits of the hash value as the index:\\
\\
\begin{fminipage}{\linewidth}
\begin{lstlisting}[]% Start your code-block

    std::uint32_t hash_value = hash.get_hash(key);
    std::uint32_t bucket_number = hash_value & (directory_size-1);
\end{lstlisting}
\end{fminipage}
\vphantom{fill}\\
Both the \verb|get| and the \verb|update| operations do not modify the structure of the buckets arrays, and is therefore completely trivial. The \verb|insert| operation is also trivial, as the \verb|push_back| function on the \verb|vector|s handles the memory management. Finally, the \verb|remove| function has to remove the wanted value, while keeping the array packed. Removing the last element of the vector can be done in $O(1)$ by invoking the \verb|pop_back| function. To exploit this for removing a arbitrary element, we copy the last element to the position of the last element, which can then be popped:\\
\\
\begin{fminipage}{\linewidth}
\begin{lstlisting}[]% Start your code-block

    keys  [i] = keys[keys.size()-1];
    values[i] = values[values.size()-1];
    keys.pop_back();
    values.pop_back();
\end{lstlisting}
\end{fminipage}
\vphantom{fill}\\
Finally, the range scan operations 
\subsubsection{Extendible hashing}
\label{subsubsec:implementation_extendible_hashing}
The extendible hashing has been implemented in the \verb|extendible_hash_table.h| file. It contains the \verb|extendible_hast_table| class, which holds the directory and buckets.

The \verb|directory| is implemented as a \verb|std::vector| of pointers to \verb|hash_bucket|s. The \verb|hash_bucket| is a \verb|struct|, which holds everything needed for one bucket. When doubling the directory, the \verb|realloc|ation is done, which increases the size of the array where it is present if possible, and reallocates the array to a different location if needed. 'An initial \verb|global_depth| has to be set through the third template parameter, which dictates how large a directory is generated upon initialization. This initial \verb|global_depth| should be set to fit an estimate of how large the directory is needed to be at any given point.' The actual directory is made double this stated size, in order to avoid unexpected growth, but there are only generated buckets for the part of the directory, which has been requested. Since the directory only contains pointers, this is not a large overhead of memory, while it makes it less likely for a costly reallocation of the entire directory to happen. \\
\\
'The \verb|hash_bucket| struct contains everything owned by a single bucket. It contains the following fields,'

\begin{lstlisting}[frame=single]  % Start your code-block

  - std::uint8_t local_depth;
  - std::uint8_t entry_count;
  - key_t*       keys;
  - value_t*     values;
\end{lstlisting}
The \verb|local_depth| is used as described previously. The \verb|entry_count| holds the amount of entries currently present in the bucket, and is used for checking whether or not the bucket is full, and for stopping probings early (described later). 

The \verb|keys| and \verb|values| have been split into two different arrays, instead of an array of structs. This was chosen, as iterations through the keys can be done without loading the values into the cache. When the correct key is found, the value can be loaded, thus only loading one cache line and thereby reducing cache-thrashing.\\
\\
\subsubsection{Partitioned Chained hashing with arrays}
\label{subsubsec:implementation_partitioned_array_hashing}

\subsubsection{Locking Primitives}
\label{subsubsec:implementation_locking_primitives}

\subsection{Thread affinity}
\label{subsec:implementation_thread_affinity}
\subsection{Throughput measurements}
\textcolor{red}{Move to experiments?}
\label{subsec:implementation_throughput_measurements}
\textcolor{red}{Move to experiments?}
\subsection{Pushing operators}
\label{subsec:implementation_pushing_operators}
\textcolor{red}{Move to implementation of hash indices?}
\subsection{Tests}
\label{subsec:implementation_tests}
\newpage

\section{Experimental setup and results}
This section presents our experimental setup, as well as the results we have found. First, the overall experimentation approach is described in Section~\ref{subsec:experiment_setup}, and the hardware on which we experiment is described in Section~\ref{subsubsec:hardware}. Next, experiments on the implemented hash functions are presented in Section~\ref{subsec:hash_func_experiments}, and finally the experiments on the implemented hash indices are presented in Section~\ref{subsec:hash_index_experiments}

\subsection{Setup}
\label{subsec:experiment_setup}
\paragraph{Hash index evaluation.} The YCSB benchmark is one of the most commonly used benchmark for performance comparisons of data-serving systems~\cite{BC10}. The benchmark standardizes many aspects of the experimental setup, such as how to generate keys and values, how to set up the used distributions, etc.. Thus using this benchmark enables an apples-to-apples comparison of data-serving systems. The YCSB benchmark also provides an extensible workload generator, called the 'client', which creates a standardized way of loading datasets and executing workloads on the data-serving systems. \\
\\
Finally, the YCSB benchmark provides six predefined core workloads, which fills out some of the evaluation space of performance trade-offs. Due to the extensibility of this benchmark, the evaluation space can be further explored, by adding new workloads, designed to test fundamental aspects of a given system, as well as domain-specific workloads.\\
\\
The experimental evaluation of the different combinations of hash functions and hash indices presented here is heavily based on the YCSB benchmark, with only a few minor modifications that adapts to the given domain. The specific implementation, as well as the modifications, is described later in this section.
\paragraph{Hash function evaluation.}
Since the project also focuses on the performance of the tabulation hashing algorithm, some key aspects of the hash functions have been evaluated, in addition to the YCSB benchmark. Specifically, we have evaluated the distributional properties and the throughput for keys of various lengths, as well as the scalability over multiple cores. This is presented first, as it yields an understanding of how the underlying hash functions perform, before they are used as part of the YCSB setup.

\subsubsection{Hardware}
\label{subsubsec:hardware}
Our primary machine has two AMD Opteron 6274 processors, each with 16 cores, running a 64-bit Linux Operating System (kernel 4.1.15-8-default). The 16 cores of each processor are distributed across two NUMA nodes. The machine has 128GiB RAM, with 64KB L1D cache, 2MB L2 cache and 64-byte cache lines. \\
\\
In order to understand the basic scalability properties of the used machine, a baseline experiment was conducted. Specifically, we wanted to see how well the machine was able to scale on a workload without any data sharing among threads. To do this, we conducted an experiment in which an increasing amount of threads perform the same work, which includes no memory allocations or loads. Specifically, we simply perform a very large amount of additions into a local variable inside each thread. Each thread is pinned to an individual core, and various patterns for thread pinning have been employed. This is discussed after the results have been presented, to help understanding what is seen.\\

Throughout multiple runs, with different sizes of work to be done, we have consistently seen results with the following baseline scalability properties, i.e. the percentage wise reduction in scalability after certain amounts of cores has been the same for all runs, yielding a graph whose shape is as seen in Figure~\ref{fig:baseline_scaling}, where we have run 10 billion additions in each thread.  

\begin{figure}[H]
  \centering
  \includegraphics[width=0.7\textwidth]{Graphs/baseline_scaling.png}
  \caption{Baseline throughput scaling of simple work}
  \label{fig:baseline_scaling}
\end{figure}
The first thing to notice is that the scaling on the first eight threads of this setup follows an ideal scaling pattern. After eight threads, the scaling is reduced, while still showing a linear pattern for eight more threads. Finally, after 16 threads, the scaling is further reduced, where using 32 cores only yield \~65\% of the throughput seen on the ideal scaling line. This yields three sections of the graph. \\

To understand each of these sections, we first have to understand how the cores for each thread have been chosen. To ensure minimal sharing, we have attempted to spread out the cores used as much as possible. Specifically, we have chosen to iteratively include a core from each NUMA node, such that all NUMA nodes have one core in use, before one NUMA node uses two cores. 

Additionally, every pair of two cores have shown to share some resources, as employing two consecutive cores yields reduced performance on both the threads running on them. Therefore, all even-numbered cores are being taken in use first, while the odd-numbered cores are being taken in use after the first 16 are included.

Finally, we have very consistently experienced that when employing a third core in a single NUMA node, the performance of all threads in the socket that contains the NUMA node is reduced. Thorough testing of this reduction have been done, employing various patterns for core orders, and the phenomenon has been present in every single case. However, we have been unable to obtain an understanding of why this happens, and have therefore simply had to accept the reduced thread scaling as the premise of the machine. Since we have four NUMA nodes in total, and only two cores from each NUMA node can be employed, a maximum of eight cores can be fully utilized, before the reduction in performance happens.\\

With this understanding of the core-employing order, the three sections of the graph can be explained. The first section corresponds to the two first cores setup for thread scaling each NUMA node, before any reduction in performance happens. The second part of the graph is the remaining eight even-numbered cores, for which the reduction from having three or more cores in each NUMA node is applied. Therefore, the scaling on this part is lower than the first part. The third part corresponds to the 16 odd-numbered cores, which have shown a slight sharing with their corresponding even-numbered core, and therefore scale slightly less than the second part. Unfortunately, we are unaware of the exact underlying causes for the seen effects.\\

Since the found results yield suboptimal baseline scaling of the system, we do not expect perfect scaling on more than eight cores. The primary reason to examine more than eight cores is therefore to see if greater reductions in performance occur, from which poor scaling can be concluded.

\subsection{Experiments on hashing algorithms}
\label{subsec:hash_func_experiments}
To evaluate the performance of the hash function implementations, three kinds of experiments have been implemented:
\begin{enumerate}[noitemsep]
  \item \emph{Distribution experiment}, which evaluates the robustness of the hash function, by evaluating the output distribution of the hashed values for different input distributions (Section~\ref{subsubsec:hash_func_dist}). 
  \item \emph{Key-Length experiment}, which evaluates the single threaded performance of the hash function, as well as how the performance is affected by the length of the key, by testing the throughput of hashing for different length of keys (Section~\ref{subsubsec:hash_func_key_len}).
  \item \emph{Multi-core experiment}, which evaluates the scalability of the hash function across multiple cores, by running the same workload as in the key-length experiment, while increasing the amount of cores used (Section~\ref{subsubsec:hash_func_multi_core}). 
\end{enumerate}

\subsubsection{Distribution experiment}
\label{subsubsec:hash_func_dist}
To evaluate the distribution of the hashed output-values, three different input-distributions have been used:
\begin{enumerate}[noitemsep]
  \item Uniform Distribution
  \item Gaussian Distribution
  \item Zipfian Distribution
\end{enumerate}

These distributions cover most seen distribution of actual data. Additionally, these distributions cover the set of distributions used by the YCSB benchmark, which includes Uniform distribution, Zipfian distribution and their custom Latest distribution, which is essentially a zipfian distribution, but where recently seen values are given the highest weight of the zipfian distribution. The gaussian distribution is simply included since it is seen so commonly in datasets from the real world.\\

For each of the three distributions, 5 million random eight-byte key-strings have been generated by generating a random 64bit integer under the given distribution, and interpreting it as a string. The random generation of integers has been done using the \verb|std::mt19937_64| implementation of the Mersenne Twister~\cite{MT02}.\\

The keys have then been hashed, and the outputs have been categorized into 256 evenly sized bins. This categorization reduces the information found, but makes it possible to visualize and therefore easily interpret the results.

\paragraph{Simple tabulation hashing.} As found by Dittrich et al., simple tabulation hashing is generally seen as a very robust hashing algorithm~\cite{RAD15}, and we therefore expect that the output distribution for each of the three input distributions is very close to uniform. \\
\\
Running the distributional experiment for the three different input distributions has yielded the results seen in Figure \ref{fig:tab_dist}.\\
\\
\begin{figure}[H]
  \centering
  \begin{subfigure}[b]{0.45\textwidth}
    \includegraphics[width=\textwidth]{Graphs/Dist/Tabulation_uniform_dist.png}
    \caption{Uniform Input Distribution}
    \label{fig:tab_dist_uni}
  \end{subfigure}
  \begin{subfigure}[b]{0.45\textwidth}
    \includegraphics[width=\textwidth]{Graphs/Dist/Tabulation_gaussian_dist.png}
    \caption{Gaussian Input Distribution}
    \label{fig:tab_dist_gauss}
  \end{subfigure}
\end{figure}
\begin{figure}[H]\ContinuedFloat
  \centering
  \begin{subfigure}[b]{0.45\textwidth}
    \includegraphics[width=\textwidth]{Graphs/Dist/Tabulation_exponential_dist.png}
    \caption{Exponential Input Distribution}
    \label{fig:tab_dist_exp}
  \end{subfigure}
  \caption{Output Distributions of Tabulation Hashing}\label{fig:tab_dist}
\end{figure}
It can be seen that for all three input distributions, the output distribution is very close to uniform, thus showing that tabulation hashing maps roughly the same amount of input elements to each hash value. This is also emphasized by the reported standard deviation, which yields a very low coefficient of variance of 0.007. Since the cost of hashing based methods increases drastically as more collisions occurs, a uniform output distribution yields the best performance. 

\paragraph{Murmur hashing.} Richter et al. also found murmur hashing to have shown strong distributional properties and being very robust~\cite{RAD15}. It is therefore also expected that murmur hashing will yield a close to uniform output-distribution for all of the three input-distributions.\\

The output distributions found from the categorized results of running the experiment using murmur hashing can be seen on the figure \ref{fig:murmur_dist}. 
\begin{figure}[H]
    \centering
    \begin{subfigure}[b]{0.45\textwidth}
        \includegraphics[width=\textwidth]{Graphs/Dist/Murmur_uniform_dist.png}
        \caption{Uniform Input Distribution}
        \label{fig:murmur_dist_uni}
    \end{subfigure}
    \begin{subfigure}[b]{0.45\textwidth}
        \includegraphics[width=\textwidth]{Graphs/Dist/Murmur_gaussian_dist.png}
        \caption{Gaussian Input Distribution}
        \label{fig:murmur_dist_gauss}
    \end{subfigure}
\end{figure}
\begin{figure}[H]\ContinuedFloat
    \centering
    \begin{subfigure}[b]{0.45\textwidth}
        \includegraphics[width=\textwidth]{Graphs/Dist/Murmur_exponential_dist.png}
        \caption{Exponential Input Distribution}
        \label{fig:murmur_dist_exp}
    \end{subfigure}
    \caption{Output Distributions of Murmur Hashing}\label{fig:murmur_dist}
\end{figure}
As seen, the output distributions of murmur hashing appear to be equally well distributed as the output distributions for tabulation hashing, which is also emphasized by the standard deviations that yield a low coefficient of variance of 0.007, equal to the coefficient of variance found for tabulation hashing.\\

Since both hash functions produce equally good output distributions, approximately the same amount of collision should occur. Thus, when using these in combination with a given hash index, the results should not be affected by one scenario having more collisions than the other. Therefore, if keeping all other factors besides the hash functions fixed, any difference in such results would be caused by one hash function having a higher throughput in the given scenario.

\subsubsection{Key-length experiment}
\label{subsubsec:hash_func_key_len}
To have a baseline of how well the two hash functions perform, a single threaded evaluation of the performance on various key lengths has been done, by hashing keys of different length and calculating the average runtime of each hash value computation. This experiment shows how well the implementation scales over key lengths, when used without any other structures. \\

Since one of the key concepts of tabulation hashing is to utilize rapid lookups in fast cache, we have tried to use the L1 cache as little as possible for the generated keys, by only generating a small amount of strings (i.e., 5) for each of the lengths from 1-64 bytes, yielding 320 strings in total. For each of the 64 string lengths, the five strings have been hashed 10000 times, in order to make the total runtime long enough for the measurement to be stable. The duration of the repeated hash value computations have been measured using a high resolution clock that yields the total amount of nanoseconds used~\cite{chrono}. \\
\\
This entire setup has then been run 10000 times yielding a total of 100 million hash value computations of each string, from which average runtime and standard deviation can be calculated for each of the 64 lengths. \\
\\
Since tabulation hashing can be specialized for a given maximum key length, by tuning the amount of tables, this would be interesting to include in the experiment. Therefore, we have run four separate runs for the tabulation hash function, using 8, 16, 32 and 64 tabulation tables, respectively. It should be noted that since tabulation hashing cannot hash keys longer than the amount of tabulation tables generated, each of the datasets only include data points up to the amount of tabulation tables employed. Only one run has been done for the murmur hash function, since it does not contain any tuning parameters. \\
\\
This experiment is interesting, as it shows an upper bound on how well the two hash functions can perform, but since hash functions are rarely used alone, it does not show the entire picture. For instance, using tabulation hash with a large index structure might contest the cache more, thus making more cache misses happen and therefore causing a lower throughput. \\
\\
The average runtime for each length can be seen in Figure \ref{fig:key_length}, which contains five curves, four for the four tabulation hashing runs using 8, 16, 32 and 64 tabulation tables, respectively, and one for the murmur hashing run. 

\begin{figure}[H]
  \centering
  \includegraphics[width=0.7\textwidth]{Graphs/Length/4_Tabulation_64_tables_length.png}\\
  \caption{Runtime of hash function invocations with different length input string.}\label{fig:key_length}
\end{figure}
\noindent
By first looking at the four data sets for the tabulation hash runs, we see that all of them experience a similar, close to linear degradation for short key lengths (i.e., up to 16 bytes). They all show the same runtime for hashing strings of just one byte, which takes roughly 6ns, while each additional byte in the string adds roughly 1.5ns.\\
\\
However, for the two runs using 32 and 64 tabulation tables respectively, we see that the scaling becomes less optimal for longer strings. Specifically, both of the two curves still show close to linear scaling for strings longer than 16 bytes, but with a steeper slope of roughly 2-2.5ns per additional byte. \\
\\
We believe that the steeper slope can be explained by looking at what the additional bytes cause to happen in the implementation. Since each byte causes a lookup in its own table, using longer strings will cause lookups in more tables. This in turn causes more cache lines to be loaded into the cache, which leads to more contention and therefore more cache misses. The overhead caused by this is however quite small.\\
\\
Secondly, it should be noted that while the four runs for tabulation hashing all appear close to linear, the standard deviation (seen on the error bars) increases as the amount of tables is increased, as well as when the length of the strings is increased. We believe this to be caused by the same contention of the cache as before.\\
\\
Finally, it can easily be seen that while the tabulation hashing has a slightly better performance on very short strings (i.e., up to three bytes), the murmur hash function scales better on the length of the key, since every four additional bytes in the key only increases the runtime of each hash function invocation by roughly 2ns: \\
\\
Murmur hashing takes \~10ns to hash a key of length one byte, and spends the same amount of time hashing strings of length up to four bytes. The increase in runtime thus only appears to happen after every fourth additional byte. This is caused by that the body loop of the murmur hashing implementation is iterated once for every 32-bit chunk of the key. It can thus be concluded that the increase in runtime primarily comes from additional iterations of this loop, while the difference between tails of length 0-3 bytes does not affect much.

\subsubsection{Multi-core experiment}
\label{subsubsec:hash_func_multi_core}
To evaluate the scalability across multiple cores, we have created an experimental setup similar to the key-length experiment (Figure~\ref{fig:key_length}), run it on multiple threads, and measured the throughput of each thread as well as the total throughput. This was done by initializing one hash function object,\footnote{Either of class \verb|tabulation_hash| or of class \verb|murmur_hash|} which is used by all the threads in their computations. The threads are all pinned to a specific core to avoid reduced performance, e.g., from potentially having more cache misses. This pinning is done as explained in Section~\ref{subsubsec:hardware}. Each thread is given its own set of five strings, to avoid sharing of these memory regions. These strings are then hashed 100 million times, from which an average total throughput can be calculated. \\
\\
Since neither of the hash function classes contain any shared state which is modified after initialization, an optimal linear increase in throughput when utilizing more cores would be expected, if external factors are disregarded. However, as we have seen on Figure~\ref{fig:baseline_scaling}, external factors are in play. \\
\\
In Figure \ref{fig:tab_cores} the average total throughput is shown for both tabulation hashing and murmur hashing. As in the key-length experiment, we have included four curves for tabulation hashing, using 8, 16, 32 and 64 tabulation tables, respectively. Since the purpose of this experiment is to show how well the throughput of each of the configurations scale over multiple cores, the length of the keys has been fixed to be the maximum length possible, which is therefore equal to the number of tabulation tables for the tabulation hash sets, and 64 for the murmur hash set. 
\\
Additionally, a linear function with slope equal to the throughput found when using just one thread has been added for each of the curves, which acts as reference for ideal scaling.\\

\begin{figure}[H]
  \centering
  \includegraphics[width=0.8\textwidth]{Graphs/Cores/4_Tabulation_8_tables_cores.png}\\
  \caption{Total throughput scalability of hash functions.}
  \label{fig:tab_cores}
\end{figure}
\noindent
While it might appear from this graph that tabulation hashing with eight tabulation tables has best overall performance, recall that this curve is generated by hashing keys of length 8, while all the other curves are generated by hashing longer keys. Similarly, while tabulation hashing with 16 tabulation tables and murmur hashing seem to have the same throughput, the former hashes keys of length 16, while the latter hashes keys of length 64. Therefore, since the goal of the graph is an understanding of the scalability of the setups, the curves should not be compared to each other.\\
\\
First note that all of the curves show three different linear sections (most apparent on the red curve), with different slopes. The first section shows very close to optimal scaling up to eight cores. In the next section (9-16 cores), the increase appears linear but with a reduced slope, thus indicating less optimal scaling. Finally, after 16 cores the scaling is reduced even further. \\
\\
To understand this behavior, recall the scalability behavior described in the hardware setup (Section~\ref{subsubsec:hardware}) on page~\pageref{subsubsec:hardware}. This scalability behavior very much resembles the behavior found with the dummy-workload, which indicates that the found scalability behavior of the hash functions is caused by the factors described there. \\
\\
Since we for different amounts of work of the dummy-workload have seen this exact behavior (i.e., 3 sections with only \~65\% of the ideal throughput found when using 32 cores), it would appear that the reduction in scalability is a percentage of the total throughput, instead of an absolute overhead. As such, a similar percentage wise reduction in performance is to be expected for these experiments. This is also emphasized by  comparing all five curves on Figure~\ref{fig:tab_cores} to their corresponding ideal scaling curve, which shows that they all exhibit 60-65\% of the ideal throughput for 32 cores.\\
\\
However, since we have no way of verifying that this is the sole cause of the reduced scaling, the primary take-away from the graph is the linear close to optimal scalability of all curves for the first eight cores. Whether or not the baseline scalability properties of the setup accounts for the entire reduction of throughput after eight cores cannot be determined.

\paragraph{Configuration of hash functions for the YCSB experiments.} Based on these hash function experiments, the following conclusion can be made: Both hash functions show good distributional properties, that should not have a negative impact on the performance of the hash indices. They are both able to handle variable length keys, but while both hash functions have comparably low runtime for short keys, the runtime for longer keys up to 64 bytes has been shown to be superior for murmur hashing. We expect this trend to extend beyond 64 bytes, as we have no indication of circumstances that changes this. \\
\\
It has, however, also been shown that using more tabulation tables does not increase the runtime of hashing short keys, compared to instantiating just enough tabulation tables. Such increase in runtime from using more tabulation tables is only seen when these tables are actually used, thus for longer keys. Additionally, we have seen similar scalability properties for all the tested configurations of hash functions. Tabulation hashing is thus configured to use 64 tables in the YCSB experiments, yielding the highest degree of flexibility, while murmur hashing is used as it is.

\subsection{Experiments on hash indices}
\label{subsec:hash_index_experiments}
This section is structured as following: Our implementation of the YCSB benchmark is described in Section~\ref{subsubsec:hash_index_ycsb_implementation}, and the primary differences to the YCSB benchmark is described and discussed in Section~\ref{subsubsec:key_diff_ycsb}. Next, the experiments on how to tune the hash indices is presented and discussed in Section~\ref{subsubsec:tuning_experiments}. Finally, the results of the running our YCSB implementation on the tuned hash indices is presented and discussed in Section~\ref{subsubsec:ycsb_results}.\\

\subsubsection{YCSB implementation}
\label{subsubsec:hash_index_ycsb_implementation}
Our YCSB implementation is built around two primary classes, namely the \verb|workload| and \verb|client| classes.\\

The \verb|workload| class handles the standardization, which yields the apples-to-apples comparison foundation. This standardization is done through standardized generation of both data and the operations making up the workload. Additionally, the \verb|workload| class handles the interaction with the database interaction layer (\verb|abstract_index|, described in Section~\ref{subsubsec:design_abstract_hash_index} on page~\pageref{subsubsec:design_abstract_hash_index}), and it should thus be seen as a replacement for the default \verb|CoreWorkload| implementation of the workload executor of the YCSB benchmark. The generation is standardized through a set of \verb|workload_properties|, which define the work to be performed, independently of any database-specific information. A class diagram of the \verb|workload| class can be seen in Figure~\ref{fig:UML_workload}.\\

\begin{figure}[H]
  \centering
  \makebox[\textwidth][c]{\includegraphics[width=\textwidth]{UML/Workload_simplified.png}}\\
  \caption{Class diagram for the YCSB workload implementation.}\label{fig:UML_workload}
\end{figure}

The \verb|client| class works on top of the \verb|workload| class, by handling the distribution of work to be done among client threads. It spawns the wanted amount of threads and instantiates a \verb|workload| for each of the threads to use. The threads each perform a sequential series of operations, in order to either load or perform transactions on the database system, which they generate using their \verb|workload| instance. The loading and transactions are done by invoking the \verb|workload| class to interact with the database interface. A class diagram of the \verb|client| class can be seen in Figure~\ref{fig:UML_client}. \\

\begin{figure}[H]
  \makebox[\textwidth][c]{\includegraphics[width=\textwidth]{UML/Client_simplified.png}}\\
  \caption{Class diagram for the YCSB client implementation.}\label{fig:UML_client}
\end{figure}

As seen, the \verb|client| is instantiated with a database implementation (anything that adheres to the \verb|abstract_index| interface), as well as a set of \verb|workload_properties|, which defines the workload to be executed. The \verb|workload_properties| struct thus controls the parameters to be standardized, such as read/write proportions, key and value maximum lengths and which distributions to use for generating the data. The six core workloads are simply specific instantiations of this struct, which makes extending the core workloads extremely easy. \\

The \verb|client| class has been implemented in the \verb=client.[h|cc]= files. It exposes the \verb|run_workload| function, which allows the workload defined by the \verb|workload_properties| to be run against the given database by a given amount of threads. This function spawns the configured number of threads, which first load the database system and then perform transactions. The threads each monitor the duration of the transaction phase. These durations are yielded to the \verb|run_workload| function, which calculates and returns the total throughput of all threads over the transaction-phase. 

This function has also been split into two different functions, namely \verb|run_build_records| and \verb|run_transactions|, which handles the loading and transaction phases, respectively, thus yielding a higher level of control over the two phases. \\

The \verb|workload| class has been implemented in the \verb=workload.[h|cc]= files. The two main functions exposed are the \verb|do_insert| and \verb|do_transaction| function, which both take an \verb|abstract_index| as parameter, and perform a loading or transaction operation on the database, respectively. Besides these two functions, only the two functions \verb|get_record_count| and \verb|get_operation_count|, which yield the amount of loading operations and transaction operations to be performed, respectively. \\

The \verb|do_insert| and \verb|do_transaction| functions are quite similar, as they both generate the content for a given single operation, and perform it. While \verb|do_insert| always performs an insertion operation on the database in order to load it, \verb|do_transaction| first generates an operation to perform, based on the operation distribution defined in the \verb|workload_properties|. \\

The generation of keys also differ for the two phases. For the loading phase, a simple counter generator is used, which ensures unique keys by incrementing an internal counter. For the transaction phase, the \verb|workload_properties| define a distribution, with which a key from a set of inserted keys is drawn. This distribution is either uniform, zipfian, or latest.\footnote{Their own zipfian distribution, which yields the highest weight on the last drawn value} \\

For the operations which require a value (i.e. \verb|insert|, \verb|update|, \verb|read_modify_write|), a value is generated by first uniformly generating a value length up to a maximum defined in the \verb|workload_properties|, and then iteratively adding a uniformly randomly chosen byte to an empty string until the given length is reached.\\

Finally, when the operation, key and (potentially) value have been generated, the \verb|workload| invokes the appropriate function on the database interface layer.\\
\subsubsection{Key differences from traditional YCSB}
\label{subsubsec:key_diff_ycsb} 
There are a couple of differences between our implementation and the YCSB benchmark as it was developed by B. Cooper~\cite{BC10}. \\

First of all, since the focus of this project has been the scalability of throughput of the multi-core in-memory key-value stores, we have chosen to omit tracking of latency for individual operations. Additionally, as part of the latency test, the original YCSB benchmark enables threads to throttle the throughput, in order to test how good the latency is under certain load pressures. This has therefore also been omitted. \\

Secondly, the original implementation is targeted to database systems in which a record can hold multiple fields. Our implementations of key-value stores only allow strings as values, and as such this would have to be omitted as well. However, since more fields implies more memory pressure on the system, and since more data has to be loaded/stored, we have enabled values to have variable length. This simulates having a different amount of fields, and is only limited by the fact that you have to operate on all the fields at once, where the original implementation allows operations on subsets of the fields.\\

Thirdly, to completely avoid any sharing of memory regions between threads, we have chosen to instantiate a \verb|workload| class for each of the threads, where the original implementation only uses one. The throughput that each thread produces is measured in operations, which thus includes the generation of keys and values. Since the \verb|workload| class is responsible for all the generation, it uses random generators. Thus, if we were to use the same \verb|workload| for all threads, they would be sharing the state of the random generators, causing a potential source of contention, which is not intended. Using one \verb|workload| for each thread completely avoids any shared state, thus eliminating a source of contention.\\

Lastly, the performance of the range scan operations is expected to be very low, since they have to perform a full scan of all entries in the hash indices. To be able to get useful results, we have decided to modify core workload \verb|E| by reducing the amount of entries being loaded into the store to 10000.
\subsubsection{Tuning of hash indices for experiments}
\label{subsubsec:tuning_experiments}
We have chosen to only use the core workloads of the YCSB benchmark, and thus not use the provided extensibility to create new workloads. This was chosen since we have found the core workloads to yield a proper comparison foundation. Instead, we have identified the primary tuning parameters in the hash indices and performed experiments with these, to find an understanding of the trade-offs connected to these parameters. These tuning experiments are presented first, as the results yield a proper set of parameters for the hash indices. When the hash indices have been tuned properly, the YCSB core workloads can be run to yield the best results.\\
\\
The primary tuning parameters are,
\begin{itemize}
  \item Maximum key length (tabulation tables), for tabulation hashing
  \item Directory size, for array hashing
  \item Initial global depth, for extendible hashing
  \item Prefix bits (number of partitions), for partitioned array hashing
\end{itemize}
\paragraph{Maximum key length (Tabulation Hashing).} As mentioned in Section~\ref{subsec:background_hashing_algorithms}, tabulation hashing requires a tabulation table for each character in the key. Thus, a maximum key length is implicitly defined by the amount of tabulation tables used. \\

Using less tabulation tables is expected to cause less cache lines to be loaded to the cache, and therefore yielding less contention on the L1 cache, thus causing a performance gain. The downside of using fewer tabulation tables is obviously that the maximum length on the keys is reduced.\\

Thus, the maximum key length does not present a trade-off between operation throughput and memory consumption. Rather, this parameter has to be set based on an evaluation of the specific use case, determining if longer keys will be present or not.\\

The three remaining tuning parameters all yield a trade-off between performance and memory usage. 
\paragraph{Directory size (Array Hashing).} For the directory size of array hashing, inserting the same amount of entries into a larger directory means less entries in each bucket, as there will be more buckets to distribute the entries between. A wider spread of the keys causes fewer entries in each bucket, and since all the operations traverse the bucket for a given key, fewer entries causes shorter runtimes for all of the operations, thus yielding an overall higher throughput. However, if there is no or very few entries in the buckets, increasing the directory size will not yield a performance improvement.\\

The performance gain comes as a trade-off with memory usage. As each bucket has a memory overhead from the additional pointer to the bucket, as well as the additional vectors and locking primitive, more buckets cause more memory to be used. The memory overhead for each bucket is 87B, as we have seven pointers (28B) and two allocators (8B),\footnote{One pointer to the bucket, and two vectors (keys and values), each having three pointers and an allocator~\cite{vector}.} one mutex (40B) and possibly one atomic flag (1B).\footnote{This is only included for the spinlock implementation.} Since both the key and the value of a single entry are strings, one entry could quickly amount up to as much memory. Thus, as long as the majority of the buckets are not empty once the database has been loaded up to an expected load, the memory overhead of the additional buckets is outweighed by the memory used by the actual data stored.\\

To verify this, an experiment has been conducted, in which the directory size has been varied for values between $2^4$ and $2^{20}$. Since the change in performance is caused by the lower probing time, we have chosen to run this experiment over core workload \verb|C|, as this workload includes 100\% read-operations, which are expected to have the lowest runtimes. Using core workload \verb|C| thus yields tuning parameter results under the highest throughput of all the core workloads, i.e., under the heaviest contention of each bucket. This makes the results useful for the other workloads as well, as none of them will produce more contention on the buckets, which could potentially cause less optimal performance. Similarly, we have chosen to run the test on 32 threads, as this also increases the contention on the buckets. Choosing the highest bucket contention ensures that the tuned hash index will perform optimal, even in the worst case.
\\
As for all of the core workloads, the loading phase of core workload \verb|C| inserts 100000 data entries. This can be used to estimate an average amount of data entries in each bucket for the different directory sizes. The throughput as well as the peak memory usage for different directory sizes can be seen in Figure~\ref{fig:dir_size}.

\begin{figure}[H]
  \centering
  \includegraphics[width=0.7\textwidth]{Graphs/tabulation_hash_array_hash_table_workload_c_dir_size_spinlock.png}\\
  \caption{Total throughput and memory usage for directory sizes of array hashing.}\label{fig:dir_size}
\end{figure}
By first interpreting the red curve for the throughput, we can see that relaxing the contention in the buckets certainly can yield a performance increase. However, as early as after a directory size of 8192 buckets (i.e. roughly 13 data entries in each bucket), we start to see a flattening of the throughput curve, indicating that the increase in throughput from increasing the directory size is diminishing. Also, after a directory size of 65536 (i.e, roughly 1.5 entries in each bucket on average), the performance does not change significantly, since further significant reduction in the traversal time through each bucket cannot occur.\\

Secondly, we can see that the memory usage increases significantly, as the size of the directory grows larger than 16384 buckets. The total memory increase seen from a directory size of $2^4$ to a directory size of $2^{20}$ is 90MB, very similar to the calculated overhead, which adds up to $(2^{20}-2^{4})*87B$ = 91MB.\\

Overall, we can conclude that the tuning of the directory size yields the expected trade-off between throughput and memory usage. For the YCSB experiments on array hashing, we are using a directory size of 65536 which has shown to yield close to optimal performance, while still employing low memory usage.
\paragraph{Initial global depth (extendible hashing).} The initial global depth of extendible hashing is quite similar to the directory size of array hashing, but with two key differences. 

First of all, the global depth defined is only valid initially, as the global depth can increase as part of insert operations. As such, poor choice of this tuning parameter can eventually be compensated for by the algorithm itself. Secondly, since the buckets have a fixed size in extendible hashing, there is a fixed upper limit on the traversal time. There is therefore never much contention in any buckets, which means that performance gain on the traversal times will be less significant.\\

The primary advantage of choosing a proper value for the initial global depth comes from avoiding the slow path of the insert algorithm (as described in Section~\ref{subsubsec:design_extendible_hashing_locking} on page~\pageref{subsubsec:design_extendible_hashing_locking}), as a larger initial global depth makes splits of the directory less frequent. Since the slow path requires exclusive access to the entire directory (and thus the entire data structure), reducing the amount of runs through this portion of the code yields an increase in the insertion throughput. However, as we saw in Section~\ref{subsubsec:design_extendible_hashing_locking}, there is a low upper bound on the maximum amount of times the slow path can be taken. So, this performance increase will be quite insignificant over the course of many operations.\\

A secondary performance increase in the insert operation comes from the fact that a bucket is instantiated for every entry in the preallocated directory. Thus, a higher initial global depth causes more buckets to be instantiated initially. These buckets will thus not have to be instantiated during the execution of actual work. The allocation of buckets is, however, not very expensive, causing also this performance increase to be insignificant. \\

The bucket allocation scheme also means, however, that the initial memory overhead of an increased preallocation of the directory is just as large for extendible hashing as for array hashing. While a doubling of the directory as part of the slow path of an insertion operation only creates new pointers to the original buckets, the number of buckets initially generated is equal to the size of the directory, and thus directly linked to the initial global depth. \\

Since the focus of the YCSB benchmark is evaluating the transaction phase, we have chosen to show what impact the initial global depth has on this phase. Since the performance increase is expected to be caused by the insert operation, we have designed a new workload that is completely similar to core workload C, except that it includes 100\% inserts. This makes the performance effect as visible as possible. However, since the loading phase causes a high number of inserts, most of the reduced throughput from a poor tuning of the initial global depth is expected to happen in this phase, and thus not included in the results.\\

The resulting throughput and memory usage of running this experiment can be seen in Figure \ref{fig:global_depht}.

\begin{figure}[H]
  \centering
  \includegraphics[width=0.7\textwidth]{Graphs/murmur_hash_extendible_hash_table_workload_insert_global_depth_mutex_murmur.png}\\
  \caption{Total throughput and memory usage for various initial global depths of extendible hashing.}\label{fig:global_depht}
\end{figure}

Notice that even though the experiment has been designed to make the performance gain as visible as possible, we do not see any performance increase from the increased preallocation caused by a greater initial global depth. As mentioned, we believe this to be due to the fact that the compensation for the poor initial allocation takes place in the loading phase.\\

Therefore,t he choice of this tuning parameter does not show any significant effect. To make the comparison of the hash indices as adequate as possible, we have chosen an initial global depth of 16, which yields a directory size of 65536 buckets, thus equal to the directory size used for the array hashing tuning. As seen in the memory curve, the initial memory usage for this configuration is almost as low as for any smaller initial global depth.

\paragraph{Prefix bits (partitioned array hashing).} For the partitioned array hashing, the primary tuning parameter is how many prefix bits of the key should be used for radix partitioning. This parameter directly dictates how many partitions are created, which in turn dictates the number of entries in each partition. A lower load on each partition yields a performance increase on all operations. For the \verb|get|, \verb|insert| and \verb|update| operations, fewer entries in the used partition yields a lower traversal time through the bucket. For the range scan operation, the possibility of elimination of some partitions (as described in Section~\ref{subsec:_design_partition_hash_index} on page~\pageref{subsec:_design_partition_hash_index}) reduces the number of entries to be searched. \\

When the number of partitions is increased, the load in each partition decreases. The lower load can potentially lead to a situation in which smaller directories for each of the partitions would be optimal. This is especially the case for range scans, where having many large partitions causes more buckets to be evaluated. If the load in each partition is so low that many of the buckets are empty, the performance of the partitioned range scan could be lower than the performance of using just a single partition.\\

Varying both the number of prefix bits and the directory size of the partitions would be a heterogeneous tuning test, which is out of the scope of this project, due to time constraints. We therefore only present results found using the same directory size for all partitions. To make the effect of the prefix bits most significant, we have chosen to make the directories of all partitions small (i.e., eight buckets in each directory), as this makes the initial contention greater. In this way, the cost of evaluating the entries  greatly outweighs the cost of accessing the buckets. However, for the runs using more prefix bits, in which more partitions are created, the load in each bucket will be decreased, causing the cost of accessing all the buckets to be a relevant factor. The results of the experiment can be seen in Figure~\ref{fig:prefix_bits}.

\begin{figure}[H]
  \centering
  \includegraphics[width=0.7\textwidth]{Graphs/murmur_hash_partitioned_array_hash_table_workload_e_prefix_bits_spinlock.png}\\
  \caption{Total throughput and memory usage for increasing number of prefix bits of partitioned array hashing.}\label{fig:prefix_bits}
\end{figure}
As seen in this figure, for the lower ranges of prefix bits, we indeed get a performance increase when increasing the number of prefix bits and thus employing more partitions. This performance increase is caused by partitions being eliminated from the range scan. 

However, after six prefix bits, the performance starts to drop. When using six bits, we have $2^6 = 64$ partitions, each with eight buckets yielding a total of $64*8 = 512$ buckets. As mentioned previously, the number of entries loaded into the store is 100, meaning that on average, only one fifth of the buckets have a single entry. This fraction will be halved for every additional prefix bit, making the cost of accessing the buckets a growing factor. After six prefix bits, we see this effect, and since the load in each bucket cannot be reduced more, the overall performance is reduced.\\

The memory overhead of using more prefix bits can be seen not to be significant in the shown example, since the individual partitions are so small. When using 15 prefix bits, a peak memory usage of 40MB has been seen. However, the memory usage can be seen to be doubled for every additional prefix bit, making the increase in memory usage exponential. If using larger partitions, the memory footprint might become a relevant factor to consider, but in the YCSB experiments on the partitioned array hashing, the prefix bits have been targeted at obtaining the maximum performance, and therefore set to six.
\subsubsection{YCSB results}
\label{subsubsec:ycsb_results}
The space of possible configurations is large. Since we use two hash functions, three hash indices, six core workloads, and two variations of locking schemes, the total configuration space is too large to be presented here. This section therefore starts with a discussion of which subsets of this configuration space are of highest interest, and then proceeds with a presentation and discussion of the experiments and their results.\\

A general reduction of the configuration space can be made on the concurrency locking schemes. Throughout every test performed, we have found that using a spinlock (as described in Section~\ref{subsubsec:design_locking_primitives} on page~\pageref{subsubsec:design_locking_primitives}) does not affect the single core performance of any of the operations in any of the hash indices, but impacts the scalability of the different hash indices. Specifically, using the spinlock for the local directory locks yield improved scalability of throughput on all operations, while using the spinlock for the directory lock of extendible hashing yields reduced scalability of throughput on all operations. \\

This complies with the theoretical discussion of which locking primitive to use for the global and local locks presented in Section~\ref{subsubsec:design_locking_primitives}. The presented results have thus been generated using the spinlock implementation for the local bucket locks and a \verb|shared_mutex| for the directory lock.\\

While using a \verb|shared_mutex| for the directory lock is advantageous over using the spinlock implementation, the high contention on the \verb|shared_mutex| makes the costs of giving the control to the operating system very high. As seen on the comparison between the two different implementations of the directory lock in Figure~\ref{fig:extendible_lock_comparison}, the high lock contention has severe impact on the performance of both locking primitives. 

\begin{figure}[H]
  \centering
  \includegraphics[width=0.7\textwidth]{Graphs/extendible_lock_comparison_f.png}\\
  \caption{Total throughput for the two locking primitives on the directory lock, using workload F.}\label{fig:extendible_lock_comparison}
\end{figure}

The workload used here (core workload F) only contains read and update operations, and therefore never causes an exclusive access mode to the directory to be taken. The scalability of both locking primitives can be seen to be very poor. The contention on the spinlock can be seen to have the greatest impact on the performance, as the performance quickly drops below the single threaded performance. For the \verb|shared_mutex| implementation, the performance impact caused by contention also eliminates scaling of throughput on multiple cores, as the performance when using more than five cores is comparable with the single threaded performance.\\

For range scans, the critical section is longer, meaning that each operation takes longer to complete. The contention on the directory lock is therefore lower when the fraction of range scan operations is higher. Running the same experiment on core workload E, in which 95\% of the operations are range scans, the advantage of using a \verb|shared_mutex| for the directory lock is amplified, as seen in Figure~\ref{fig:extendible_lock_comparison_e}.

\begin{figure}[H]
  \centering
  \includegraphics[width=0.7\textwidth]{Graphs/extendible_lock_comparison_e.png}\\
  \caption{Total throughput for the two locking primitives on the directory lock, using workload E.}\label{fig:extendible_lock_comparison_e}
\end{figure}

In this figure, we observe some degree of scalability for extendible hashing when a \verb|shared_mutex| is used for the directory lock. The longer critical sections of the operations of this experiment causes less contention on the directory lock, since the throughput of operations will be lower. Since the longer critical sections is the main difference between the experiments of Figure~\ref{fig:extendible_lock_comparison} and Figure~\ref{fig:extendible_lock_comparison_e}, we believe that the high contention of the directory lock is the main cause of the poor scalability of extendible hashing.\\

These results show that pessimistic concurrency control overall is very unfitting for the global mutable directory of extendible hashing. However, since the best results have been found using the \verb|shared_mutex| implementation, all results shown have been created using this locking primitive for the directory lock. \\

\paragraph{YCSB core workload results.} With this narrowing of the configuration space and by plotting the results of all three hash indices on the same graphs, we end up with 12 combinations of hash functions and workload. As each of the core workloads are designed to simulate a specific application, we have found showing results for all six core workloads relevant. All 12 graphs are therefore presented in the following sections, showing the pair of graphs for the two hash functions side by side, in order to make comparison easier.\footnote{We refer to Appendix~\ref{appendix:full_result_overview} for full size graphs.}\\

In general for all six pairs of graphs, we shall soon see that the two graphs of each pair are very similar. Since the implementation of murmur hashing has shown superior throughput over the implementation of tabulation hashing, this suggests that the runtime of the hash functions is insignificant in the overall picture. The discussion on each pair of graphs therefore primarily regards the three hash indices. Additionally, since high contention on the directory lock of extendible hashing hinders any scaling, this hash index is only discussed when other interesting features show.\\

The core workloads is presented in an order that makes showing certain features of the implementation easiest, and thus not in the alphabetic order. 

\subparagraph{Core workload C.} Core workload C is a read only workload, thus consisting of 100\% reads. This workload thus yields an understanding of how well the implemented hash indices throughput scale over multiple cores, since not a single data entry is modified after the loading phase. 

The workload simulates the load received by a read-only system such as the Hadoop filesystem implementation that lets users read data from a remote Hadoop HDFS cluster~\cite{HFTP}. The results can be seen in Figure~\ref{fig:res_c}.\\
\begin{figure}[ht]
  \centering
  \begin{subfigure}[b]{0.45\textwidth}
    \includegraphics[width=\textwidth]{Graphs/murmur_hash_workload_c_spinlock_large_max.png}
    \caption[]{Murmur hashing.}
    \label{fig:mur_c}
  \end{subfigure} \hfill
  \begin{subfigure}[b]{0.45\textwidth}
    \includegraphics[width=\textwidth]{Graphs/tabulation_hash_workload_c_spinlock_large_max.png}
    \caption[]{Tabulation hashing.}
    \label{fig:tab_c}
  \end{subfigure}
  \caption[]{Throughput results for core workload C.}
  \label{fig:res_c}
\end{figure}

For both array hashing and partitioned array hashing, a close to optimal scaling can be seen for the first eight threads. Additionally, using up to 16 threads also show good scaling properties for both hash indices. For more than 16 threads, only array hashing exhibits proper scaling, while the total throughput of partitioned starts to decrease for every additional thread.
\textcolor{red}{Proper explanation for this is needed! Discuss Friday.}
\subparagraph{Core workload B and D.} 
Core workload B and D are very similar, since they both are read mostly workloads, including 95\% reads and just 5\% updates. The difference between them is the key generation distribution, which for core workload B is zipfian, while core workload D uses their \verb|latest| distribution. The only difference between the distributions is thus which keys have the highest weight, as both distributions have zipfian properties.\\

These kinds of workloads are very commonly seen, as they amongst other simulate interactions on social media, such as user updates, where every update is read by many users. The results found for the two workloads are also quite similar, and will be presented in succession in Figure~\ref{fig:res_b} and Figure~\ref{fig:res_d} and discussed together.\\
\begin{figure}[ht]
  \centering
  \begin{subfigure}[b]{0.45\textwidth}
    \includegraphics[width=\textwidth]{Graphs/murmur_hash_workload_b_spinlock_large_max.png}
    \caption[]{Murmur hashing.}
    \label{fig:mur_b}
  \end{subfigure} \hfill
  \begin{subfigure}[b]{0.45\textwidth}
    \includegraphics[width=\textwidth]{Graphs/tabulation_hash_workload_b_spinlock_large_max.png}
    \caption[]{Tabulation hashing.}
    \label{fig:tab_b}
  \end{subfigure}
  \caption[]{Throughput results for core workload B.}
  \label{fig:res_b}
\end{figure}
\begin{figure}[ht]
  \centering
  \begin{subfigure}[b]{0.45\textwidth}
    \includegraphics[width=\textwidth]{Graphs/murmur_hash_workload_d_spinlock_large_max.png}
    \caption[]{Murmur hashing.}
    \label{fig:mur_d}
  \end{subfigure} \hfill
  \begin{subfigure}[b]{0.45\textwidth}
    \includegraphics[width=\textwidth]{Graphs/tabulation_hash_workload_d_spinlock_large_max.png}
    \caption[]{Tabulation hashing.}
    \label{fig:tab_d}
  \end{subfigure}
  \caption[]{Throughput results for core workload D.}
  \label{fig:res_d}
\end{figure}

In all four cases, we see linear scaling for the first eight threads. However, in these cases the scaling is scaling is less 

\subparagraph{Core workload A.} Core workload A is an update heavy workload, in which half of the operations are updates, while the other half is reads. A session store that records recent actions is an example of an application for which such workload is common. 
\begin{figure}[ht]
  \centering
  \begin{subfigure}[b]{0.45\textwidth}
    \includegraphics[width=\textwidth]{Graphs/murmur_hash_workload_a_spinlock_large_max.png}
    \caption[]{Murmur hashing.}
    \label{fig:mur_a}
  \end{subfigure} \hfill
  \begin{subfigure}[b]{0.45\textwidth}
    \includegraphics[width=\textwidth]{Graphs/tabulation_hash_workload_a_spinlock_large_max.png}
    \caption[]{Tabulation hashing.}
    \label{fig:tab_a}
  \end{subfigure}
  \caption[]{Throughput results for core workload A.}
  \label{fig:res_a}
\end{figure}


\subparagraph{Core workload E.} 
\begin{figure}[ht]
  \centering
  \begin{subfigure}[b]{0.45\textwidth}
    \includegraphics[width=\textwidth]{Graphs/murmur_hash_workload_e_spinlock_large_max.png}
    \caption[]{Murmur hashing.}
    \label{fig:mur_e}
  \end{subfigure} \hfill
  \begin{subfigure}[b]{0.45\textwidth}
    \includegraphics[width=\textwidth]{Graphs/tabulation_hash_workload_e_spinlock_large_max.png}
    \caption[]{Tabulation hashing.}
    \label{fig:tab_e}
  \end{subfigure}
  \caption[]{Throughput results for core workload E.}
  \label{fig:res_e}
\end{figure}


\subparagraph{Core workload F.} 
\begin{figure}[ht]
  \centering
  \begin{subfigure}[b]{0.45\textwidth}
    \includegraphics[width=\textwidth]{Graphs/murmur_hash_workload_f_spinlock_large_max.png}
    \caption[]{Murmur hashing.}
    \label{fig:mur_f}
  \end{subfigure} \hfill
  \begin{subfigure}[b]{0.45\textwidth}
    \includegraphics[width=\textwidth]{Graphs/tabulation_hash_workload_f_spinlock_large_max.png}
    \caption[]{Tabulation hashing.}
    \label{fig:tab_f}
  \end{subfigure}
  \caption[]{Throughput results for core workload F.}
  \label{fig:res_f}
\end{figure}


\newpage
\section{Discussion}
further work
\begin{itemize}
  \item Finer granularity of locks (entry locks)
  \item Optimistic concurrency control
\end{itemize}
\newpage
\section{Conclusion}
\newpage

\appendix
\section{Results in full size}
\label{appendix:full_result_overview}
\begin{figure}[H]
  \centering
  \includegraphics[width=\textwidth]{Graphs/tabulation_hash_workload_a_spinlock_max.png}\\
  \caption[]{Throughput for workload A, using tabulation hashing.}\label{fig:tab_a_full}
\end{figure}
\begin{figure}[H]
  \centering
  \includegraphics[width=\textwidth]{Graphs/tabulation_hash_workload_b_spinlock_max.png}\\
  \caption[]{Throughput for workload B, using tabulation hashing.}\label{fig:tab_b_full}
\end{figure}
\begin{figure}[H]
  \centering
  \includegraphics[width=\textwidth]{Graphs/tabulation_hash_workload_c_spinlock_max.png}\\
  \caption[]{Throughput for workload C, using tabulation hashing.}\label{fig:tab_c_full}
\end{figure}
\begin{figure}[H]
  \centering
  \includegraphics[width=\textwidth]{Graphs/tabulation_hash_workload_d_spinlock_max.png}\\
  \caption[]{Throughput for workload D, using tabulation hashing.}\label{fig:tab_d_full}
\end{figure}
\begin{figure}[H]
  \centering
  \includegraphics[width=\textwidth]{Graphs/tabulation_hash_workload_e_spinlock_max.png}\\
  \caption[]{Throughput for workload E, using tabulation hashing.}\label{fig:tab_e_full}
\end{figure}
\begin{figure}[H]
  \centering
  \includegraphics[width=\textwidth]{Graphs/tabulation_hash_workload_f_spinlock_max.png}\\
  \caption[]{Throughput for workload F, using tabulation hashing.}\label{fig:tab_f_full}
\end{figure}
\begin{figure}[H]
  \centering
  \includegraphics[width=\textwidth]{Graphs/murmur_hash_workload_a_spinlock_max.png}\\
  \caption[]{Throughput for workload A, using murmur hashing.}\label{fig:mur_a_full}
\end{figure}
\begin{figure}[H]
  \centering
  \includegraphics[width=\textwidth]{Graphs/murmur_hash_workload_b_spinlock_max.png}\\
  \caption[]{Throughput for workload B, using murmur hashing.}\label{fig:mur_b_full}
\end{figure}
\begin{figure}[H]
  \centering
  \includegraphics[width=\textwidth]{Graphs/murmur_hash_workload_c_spinlock_max.png}\\
  \caption[]{Throughput for workload C, using murmur hashing.}\label{fig:mur_c_full}
\end{figure}
\begin{figure}[H]
  \centering
  \includegraphics[width=\textwidth]{Graphs/murmur_hash_workload_d_spinlock_max.png}\\
  \caption[]{Throughput for workload D, using murmur hashing.}\label{fig:mur_d_full}
\end{figure}
\begin{figure}[H]
  \centering
  \includegraphics[width=\textwidth]{Graphs/murmur_hash_workload_e_spinlock_max.png}\\
  \caption[]{Throughput for workload E, using murmur hashing.}\label{fig:mur_e_full}
\end{figure}
\begin{figure}[H]
  \centering
  \includegraphics[width=\textwidth]{Graphs/murmur_hash_workload_f_spinlock_max.png}\\
  \caption[]{Throughput for workload F, using murmur hashing.}\label{fig:mur_f_full}
\end{figure}
\newpage


\begin{thebibliography}{99}

\bibitem{MT12}
 Mao, Y.; Kohler, E.; Robert, M. (2012) \\
 \emph{Cache Craftiness for Fast Multicore Key-Value Storage}\\
 http://doi.acm.org/10.1145/2168836.2168855

\bibitem{SILO13}
 Tu, S.; Zheng, W.; Kohler, E.; Liskov, B.; Madden, S. (2013) \\
 \emph{Speedy Transactions in Multicore In-Memory Databases} \\
 http://doi.acm.org/10.1145/2517349.2522713

\bibitem{Zobrist}
 Zobrist, A. (1970)\\
 \emph{A New Hasahing Method with Application for Game Playing.}\\
 http://research.cs.wisc.edu/techreports/1970/TR88.pdf

\bibitem{WC79}
 Carter, J.; Wegman, Mark. (1979)\\
 \emph{Universal classes of hash functions.}\\
 Journal of Computer and System Sciences, 143-154:\\
 https://www.cs.princeton.edu/courses/archive/fall09/cos521/Handouts/universalclasses.pdf

\bibitem{TZ09}
 Thorup, M.; Zhang, Y. (2009)\\
 \emph{Tabulation based 5-universal hashing and linear probing.}\\
 In Proc. 12th Workshop on Algorithm Engineering and Experiments (ALENEX), 2009

\bibitem{PT11}
 Pătraşcu, M.; Thorup, M. (2011)\\
 \emph{The Power of Simple Tabulation Hashing}\\
 http://arxiv.org/abs/1011.5200

\bibitem{RAD15} % 0
 Richter, S.; Alvarez, V.; Dittrich, J. (2015)\\
 \emph{A seven Dimensional Analysis of Hashing Methods and its Implications on Query Processing}\\
 http://www.vldb.org/pvldb/vol9/p96-richter.pdf

% Hash Tables
\bibitem{ItA09}
 Cormen, Thomas H.; Leiserson, Charles E.; Rivest, Ronald L.; Stein, Clifford (2009). \\
 \emph{Introduction to Algorithms (3rd ed.).} \\
 Massachusetts Institute of Technology. pp. 253–280. ISBN 978-0-262-03384-8.

\bibitem{NA09} % 1
 Askitis, N. (2009)\\
 \emph{Fast and Compact Hash Tables for Integer Keys}\\
 http://dl.acm.org/citation.cfm?id=1862675

\bibitem{AJ05}
 Askitis, N.; Zobel, J. (2005)\\
 \emph{Cache-Conscious Resolution in String Hash Tables}
 http://goanna.cs.rmit.edu.au/~jz/fulltext/spire05.pdf

\bibitem{ADSJ13} % 6
 Dudás, Á.; Juhasz, S. (2013)\\
 \emph{Blocking and non-blocking concurrent hash tables in multi-core systems}\\
 http://www.wseas.org/multimedia/journals/computers/2013/5705-125.pdf

\bibitem{BC10}
 Cooper, B. F.; Silberstein, A.; Tam, E.; Ramakrishnan, R.; Sears, R. (2010)\\
 \emph{Benchmarking Cloud Serving Systems with YCSB}\\
 https://www.cs.duke.edu/courses/fall13/cps296.4/838-CloudPapers/ycsb.pdf

\bibitem{Mur3}
 Appleby, A.\\
 \emph{Original Murmurhash3 implementation}\\
 https://github.com/aappleby/smhasher, version 08/01/16

\bibitem{DPH90}
 Dietzfelbinger, Martin, et al. (1994)\\
 \emph{Dynamic perfect hashing: Upper and lower bounds.}\\
 SIAM Journal on Computing 23.4 (1994): 738-761.

\bibitem{nphf}
 \emph{Napoleon machine, for the HIPERFIT research center}\\
 http://napoleon.hiperfit.dk/

\bibitem{dms03}
 Ramakrishnan, Raghu.; Gehrke, J. (2003)\\
 \emph{Database Management Systems, Extendible Hashing}\\
 Database Management Systems, 3rd edition: 738-761.

\bibitem{radix}
 Polychroniou, O.; Ross, K. (2014)\\
 \emph{A Comprehensive Study of Main-Memory Partitioning and its Application to Large-Scale Comparison- and Radix-Sort}\\
 http://www.cs.columbia.edu/~orestis/sigmod14I.pdf

\bibitem{ARTful}
 Leis, V.; Kemper, A.; Neumann, T. (2013)\\
 \emph{The Adaptive Radix Tree: ARTful Indexing for Main-Memory Databases}\\
 http://dx.doi.org/10.1109/ICDE.2013.6544812

\bibitem{silt11}
 Lim, H.; Fan, B.; Andersen, D.; Kaminsky, M. (2011)\\
 \emph{SILT: A Memory-Efficient, High-Performance Key-Value Store}\\
 https://www.cs.cmu.edu/~dga/papers/silt-sosp2011.pdf

\bibitem{cassandra}
 \emph{Apache Cassandra Database}\\
 http://cassandra.apache.org/

\bibitem{redis}
 \emph{Redis data structure store}\\
 http://redis.io/

\bibitem{oracle}
 \emph{Oracle NoSQL Database}\\
 http://www.oracle.com/technetwork/database/database-technologies/nosqldb/overview/index.html

\bibitem{hbase}
 \emph{Apache HBase Hadoop Database} \\
 https://hbase.apache.org/

\bibitem{dynamo}
 DeCandia, G. et al. (2007)\\
 \emph{Dynamo: Amazon’s Highly Available Key-value Store}\\
 http://www.allthingsdistributed.com/files/amazon-dynamo-sosp2007.pdf

\bibitem{HFTP}
 \emph{Hadoop filesystem implementation}\\
 https://hadoop.apache.org/docs/r1.2.1/hftp.html

\bibitem{MT02}
 Matsumoto, M.; Nishimura, T. (2002) \\
 \emph{Mersenne Twister Home Page} \\
 http://www.math.sci.hiroshima-u.ac.jp/~m-mat/MT/emt.html

\bibitem{chrono}
 \emph{Chrono high resolution clock} \\
 http://en.cppreference.com/w/cpp/chrono/high\_resolution\_clock

\bibitem{vector}
\emph{Vector implementation used}
http://en.cppreference.com/w/cpp/container/vector

\bibitem{spinlock}
 Boddaert, G. \\
 \emph{User-Level Spin Locks} \\
 http://www.codeproject.com/Articles/784/User-Level-Spin-Locks
\end{thebibliography}
\end{document}
